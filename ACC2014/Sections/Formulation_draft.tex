
\documentclass[12pt]{amsart}
\usepackage{geometry} % see geometry.pdf on how to lay out the page. There's lots.
\geometry{a4paper} % or letter or a5paper or ... etc
% \geometry{landscape} % rotated page geometry

\newtheorem{definition}{Definition}
\newtheorem{prop}{Proposition}
\newtheorem{cor}{Corollary}


% See the ``Article customise'' template for come common customisations

\title{}
\author{}
\date{} % delete this line to display the current date

%%% BEGIN DOCUMENT
\begin{document}

\section{Problem Formulation}\label{sec:Formulation}

\begin{enumerate}
	\item Define our system and the problem under consideration
	\item Define our half density advection equation
	\item Formalize the relationship between half density advection equation and trajectories of the system
	\item Formalize properties of the advection equation (like mass preserving and anti-hermitian)
\end{enumerate}

Let $M$ be a volume manifold of dimension $n$ with volume form $\mu \in \Omega^n(M)$.
\begin{definition}
	Let $F(M)$ denote the frame-bundle of $M$.  A density is a function $\nu : F(M) \to \mathbb{R}$ such that
	\[
		\nu( Ae) = |\det(A)| \nu(e)	
	\]
	where $e = (e_1, \dots , e_n)$, $A \in \mathbb{R}^{n \times n}$, and $A e = (A^1_j e_j , \dots , A^i_j e_j , \dots , A^n_j e_j )$.
	A \emph{half-density} is a funcion $\Psi: F(M) \to \mathbb{C}$ such that $\Psi(Ae) = |\det(A)|^{1/2} \Psi(e)$.
	We denote the space of half-densities by $| \bigwedge |^{1/2}(M)$.
\end{definition}
  Let $\Psi_1$ and $\Psi_2$ be half-densities.  Then we observe that $\Psi_1(Ae)^{\ast} \Psi_2(Ae) = | \det(A) | \Psi_1(e)^{\ast} \Psi_2(e)$.
  Thus $\Psi_1^{\ast}( \cdot) \Psi_2(\cdot)$ is a complex valued density, which we may integrator over $M$.
  This yields the complex inner product $\langle \Psi_1 , \Psi_2 \rangle = \int_M \Psi_1^{\ast} \Psi_2$.
  We call the completion of $| \bigwedge |^{1/2}(M)$ under the norm induced by $\langle \cdot , \cdot \rangle$ the Hilbert space of half-densities, which we denote by $\mathcal{H}$.
 
  ...
 
 The equation of motion is
 \begin{align}
 	\partial_t \psi = \pounds_{X}[\psi] + \frac{1}{2} {\rm div}(X) \psi \label{eq:advection}
 \end{align}
 
  \begin{prop}
  	Let $\Psi = \psi \cdot \sqrt{\mu}$ be a time dependent half-density.  If $\psi$ satisfies \eqref{eq:advection}, then $\rho = \| \psi \|^2$ satisfies the Louiville equation.  Conversely, if $\rho$ satisfies the Louiville equation, then there exists a $\psi$ such that $\rho = \| \psi \|^2$ which satisfies the Louiville equation.
  \end{prop}
  
  \begin{cor}
  	Let $x( \cdot )$ is an orbit of $X$ originating from $x_0 \in M$ if and only if there exists a sequence of time-dependent half-densities $\Psi_1(t) , \Psi_2(t), \dots$ satisfying \eqref{eq:advection} and such that the density $\mu_n(0) = \Psi_n^{\ast}(0) \Psi_n(0)$ converges to a Dirac-delta function centered at $x_0$.
  \end{cor}
  
  Properties to formalize.  I think we have a group homomorphism, mass-preservation=unitarity = anti-hermetian infinitesimal generators.

\end{document}