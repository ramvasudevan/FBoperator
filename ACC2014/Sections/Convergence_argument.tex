\documentclass[12pt]{amsart}
\usepackage{geometry} % see geometry.pdf on how to lay out the page. There's lots.
\geometry{a4paper} % or letter or a5paper or ... etc
% \geometry{landscape} % rotated page geometry


\newtheorem{definition}{Definition}
\newtheorem{prop}{Proposition}
\newtheorem{corollary}{Corollary}
\newtheorem{lemma}{Lemma}

\newcommand{\R}{\mathbb{R}}
\newcommand{\N}{\mathbb{N}}
\newcommand{\card}[1]{{\lvert#1\rvert}}

\newcommand{\abs}[1]{\ensuremath{\left\lvert #1\right\rvert}}
\newcommand{\absb}[1]{\ensuremath{\bigl\lvert #1\bigr\rvert}}
\newcommand{\absB}[1]{\ensuremath{\Bigl\lvert #1\Bigr\rvert}}
\newcommand{\absbb}[1]{\ensuremath{\biggl\lvert #1\biggr\rvert}}
\newcommand{\absBB}[1]{\ensuremath{\Biggl\lvert #1\Biggr\rvert}}

\newcommand{\norm}[1]{\ensuremath{\left\lVert #1\right\rVert}}
\newcommand{\normb}[1]{\ensuremath{\bigl\lVert #1\bigr\rVert}}
\newcommand{\normB}[1]{\ensuremath{\Bigl\lVert #1\Bigr\rVert}}
\newcommand{\normbb}[1]{\ensuremath{\biggl\lVert #1\biggr\rVert}}
\newcommand{\normBB}[1]{\ensuremath{\Biggl\lVert #1\Biggr\rVert}}

\newcommand{\set}[1]{\ensuremath{\left\{ #1\right\}}}
\newcommand{\setb}[1]{\ensuremath{\bigl\{ #1\bigr\}}}
\newcommand{\setB}[1]{\ensuremath{\Bigl\{ #1\Bigr\}}}
\newcommand{\setbb}[1]{\ensuremath{\biggl\{ #1\biggr\}}}
\newcommand{\setBB}[1]{\ensuremath{\Biggl\{ #1\Biggr\}}}

\newcommand{\paren}[1]{\!\ensuremath{\left( #1\right)}}
\newcommand{\parenb}[1]{\ensuremath{\bigl( #1\bigr)}}
\newcommand{\parenB}[1]{\ensuremath{\Bigl( #1\Bigr)}}
\newcommand{\parenbb}[1]{\ensuremath{\biggl( #1\biggr)}}
\newcommand{\parenBB}[1]{\ensuremath{\Biggl( #1\Biggr)}}

\newcommand{\sqparen}[1]{\!\ensuremath{\left[ #1\right]}}
\newcommand{\sqparenb}[1]{\ensuremath{\bigl[ #1\bigr]}}
\newcommand{\sqparenB}[1]{\ensuremath{\Bigl[ #1\Bigr]}}
\newcommand{\sqparenbb}[1]{\ensuremath{\biggl[ #1\biggr]}}
\newcommand{\sqparenBB}[1]{\ensuremath{\Biggl[ #1\Biggr]}}


% See the ``Article customise'' template for come common customisations

\title{}
\author{}
\date{} % delete this line to display the current date

%%% BEGIN DOCUMENT
\begin{document}

\section{Convergence Argument}

Let $M$ be a volume manifold of dimension $d$ and consider the case of a partial differential operator $X$ where $X f(x) = \sum_{\abs{m}\leq 1} a_m(x) \partial^m f(x)$ with coefficients $a_m \in C^{\infty}(M)$. Assume that $(\psi_\lambda)_{\lambda \in \nabla}$ and $(\tilde{\psi}_\lambda)_{\lambda \in \nabla}$ are a pair of wavelet bases. We denote by $P_j$ the projection onto the scaling functions associated with $\psi$ of resolution $2^{-j}$. We also let $\card{\lambda}$ denote the resolution level of the index $\lambda$. 

Define weighted spaces as:
\begin{equation}
	l_t^2(\nabla) = \left\{ (c_\lambda)_{\lambda\in\nabla} \mid \norm{(c_\lambda)}_{l^2_t}^2 = \sum_{\lambda \in \nabla} 2^{2t\card{\lambda}}\abs{c_\lambda}^2 < \infty  \right\}.
\end{equation}
When $t = 0$ we write $l^2$ for the aforementioned weighted space. We are interested in evaluating the entries:
\begin{equation}
	m_{\lambda,\mu} = \langle X \psi_\lambda, \tilde{\psi}_\mu \rangle.
\end{equation}
We let $M=(m_{\lambda,\mu})_{\lambda,\mu \in \nabla}$. Assume further that there exists an $\alpha,n \in \N$ such that the wavelets are in $C^{\alpha}$, $\alpha \geq \frac{d}{2} + 1$, and that the wavelets are orthogonal to polynomials of degree $n \geq \alpha - 1$.  Note by Theorem 3.6.1 in \cite{} the norm equivalence between elements in $W^{t,2}$ and the $l_t^2$ norm of the wavelet decomposition of that same element using either wavelet bases for $0\leq t \leq \alpha$. For the sake of convenience, we let $s = \frac{d}{2} + 1$. 

\begin{lemma}
	Let $x \in W^{s,2}(M)$, then $\norm{x - P_j x}_{W^{s,2}} \leq  2^{-j(\alpha - s)}\norm{x}_{W^{\alpha,2}}$,
\end{lemma}
\begin{proof}
	This follows from either Theorem 3.3.2 or 3.10.2 in \cite{}.
\end{proof}

\begin{lemma}
	$\abs{m_{\lambda,\mu}} \leq C 2^{-(\frac{d}{2}+\frac{1}{2})\abs{\abs{\lambda}-\abs{\mu}}}2^{\frac{1}{2}(\abs{\lambda}+\abs{\mu})}$
\end{lemma}
\begin{proof}
	\begin{align}
		\abs{m_{\lambda,\mu}} 	&= \int \abs{X \psi_{\lambda}(x) \tilde{\psi}_{\mu}(x)} dx \\
								&\leq \norm{\tilde{\psi}_\mu}_{L^1}\inf_{g \in \Pi_n} \norm{X \psi_{\lambda} - g}_{L^{\infty}} \\
								&\leq C_1 2^{-\card{\mu}\frac{d}{2}}\norm{X\psi_{\lambda}}_{L^{\infty}} \\
								&\leq C_2 2^{-\card{\mu}\frac{d}{2}}\sum_{\card{m}\leq 1}\norm{\psi_{\lambda}^{(m)}(x)}_{L^{\infty}} \\
								&\leq C_3 2^{-\card{\mu}\frac{d}{2}} 2^{\card{\lambda}\frac{d}{2}} \sum_{\card{m}\leq 1} (2^{\card{\lambda}})^m \\
								&\leq C_3 d 2^{-\card{\mu}\frac{d}{2}} 2^{\card{\lambda}(1+\frac{d}{2})} \\
								&= C_3 d 2^{d/2(\card{\lambda} - \card{\mu})}2^{\card{\lambda}},
	\end{align}
	where the second line follows from the fact that $\tilde{\psi}$ is orthogonal to polynomials of degree $n$, the third line follows from Theorems 3.3.1 and 3.10.2 in \cite{}, the fourth line follows from the definition of $X$, and the fifth follows from the definition of the wavelets. The desired result follows from a symmetry argument and a rearrangement.
\end{proof}

\begin{lemma}
	$M$ is a bounded linear operator from $W^{s,2}(M)$ to $L^2(M)$.
\end{lemma}
\begin{proof}
	Consider $x \in W^{s,2}(M)$ with wavelet expansion $(x_\lambda)_{\lambda \in \nabla}$ such that $\norm{(x_\lambda)}_{l_s^2} = 1$, then:
	\begin{align}
		\norm{Mx}_{l^2}^2   &=    \sum_{\mu \in \nabla} \abs{ \sum_{\lambda \in \nabla} m_{\mu,\lambda} x_{\lambda} }^2 \\
							&\leq C\sum_{\mu \in \nabla} \sum_{\lambda \in \nabla} 2^{-(d+1)\abs{\card{\lambda} - \card{\mu}}} 2^{\card{\lambda}+\card{\mu}} \abs{x_\lambda}^2 \\
							&\leq C\sum_{\lambda \in \nabla} 2^{\card{\lambda}} \abs{x_\lambda}^2 \sum_{\mu \in \nabla} 2^{-(d+1)\abs{\card{\lambda}-\card{\mu}}}2^\mu \\
							&\leq C\sum_{\lambda \in \nabla}  2^{\card{\lambda}} \abs{x_\lambda}^2 \left( \sum_{0\leq\mu\leq\lambda} 2^{-(d+1)(\lambda - \mu)}2^{\mu} + \sum_{\mu > \lambda} 2^{-(d+1)(\mu-\lambda)}2^{\mu} \right) \\
							&\leq C(2^{(d+2)}+2^{-d}) \sum_{\lambda \in \nabla} 2^{2\lambda} \abs{x_{\lambda}}^2 \\
							&\leq C(2^{(d+2)}+2^{-d}).
	\end{align}
\end{proof}

\begin{lemma}
	Let $M^j=(m^j_{\lambda,\mu})_{\lambda,\mu \in \nabla}$ be such that $m^j_{\lambda,\mu} = m_{\lambda,\mu}$ if $\card{\lambda}$ and $\card{\mu}$ are both less than or equal to $j$ and $0$ otherwise. Then there exists a $C > 0$ such that for all $x \in W^{s,2}$:
	\begin{equation}
		\norm{(M-M^j)(P_jx)}_{l^2} \leq C2^{-(d/2)(j+1)} \norm{x}_{W^{s,2}}.
	\end{equation}
\end{lemma}
\begin{proof}
	\begin{align}
		\norm{(M-M^j)(P_j x)}_{l^2} &= \sum_{\lambda \in \nabla} \sum_{\card{\mu}\leq j} \abs{m_{\lambda,\mu}-m^k_{\lambda,\mu}}^2 \abs{(P_j x)_\mu}^2 \\
		&= \sum_{\card{\mu}\leq j}\abs{(P_j x)_\mu}^2 \sum_{\card{\lambda}>j}\abs{m_{\lambda,\mu}}^2 \\
		&\leq C'\sum_{\card{\mu}\leq j}\abs{(P_j x)_\mu}^2 2^{(d+2)\card{\mu}}\sum_{\card{\lambda}>j}2^{-d\card{\lambda}} \\
		&\leq C' 2^{-2(\frac{d}{2})(j+1)}\norm{P_j x}^2_{l^2_{d/2+1}}
	\end{align}
	The result follows by Remark 3.3.1 in \cite{}.
\end{proof}

\begin{corollary}
	$\norm{Mx - M^j(P_j x)}_{l^2} \leq C \left( 2^{-sj}\norm{x}_{W^{s,2}} + 2^{-\frac{d}{2}(j+1)}\norm{x}_{W^{s,2}} \right)$.
\end{corollary}
\begin{proof}
	Follows from the Triangle Inequality:
	\begin{multline}
		\norm{Mx - M P_j x + M P_j x - M^j P_j x}_{l^2} \leq \norm{M}_{l^2} \norm{x - P_j x}_{l^2} + \\ + \norm{(M-M^j)(P_jx)}_{l^2}.
	\end{multline}
\end{proof}

The only weird thing here is that we use the $L^2$--norm on both the domain and the range, but we only have the rate of convergence for points in the domain that are $W^{s,2}$.

\end{document}