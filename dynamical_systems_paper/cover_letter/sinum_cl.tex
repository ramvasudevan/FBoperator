\documentclass{letter}
\usepackage[absolute]{textpos}
\usepackage{graphicx}
\usepackage{epstopdf}
\usepackage{calligra}
\usepackage{color}

\setlength{\TPHorizModule}{30mm}
\setlength{\TPVertModule}{\TPHorizModule}
\textblockorigin{10mm}{10mm} %start textpos near top left corner
%\signature{Dr. Henry O. Jacobs}
\address{2250 GG Brown  Bldg.\\
2350 Hayward \\
Ann Arbor, MI  48109-2125 \\
734 764-2694  FAX: 734 647-9379}

\begin{document}
\begin{textblock}{1}(1,1)
  \includegraphics[width=80mm]{CoE-ME-horiz-PMS}
\end{textblock}
\begin{letter}{SIAM Journal on Numerical Analysis}
\opening{Dear Editor,}
Please find attached a manuscript that we would like to submit for
publication in the SIAM Journal on Numerical Analysis.

In the paper we present spectral discretizations for transport and advection PDEs which preserve many of the qualities of transport and advection.
For example, the operation of multiplying functions on a manifold can be viewed as bilinear operator on $C(M)$.
The evolution of the transport equation preserves this infinite dimensional operator.
Our discretization manages to conserve a discrete analog of function multiplication, and this discrete-multiplication operator converges spectrally to the traditional notion of function multiplication.
Standard spectral Galerkin and psuedo-spectral discretizations do not admite such a conservation law.
This is merely one of many conservation laws addressed in the paper.

These preservation properties are tantamount to achieving stable numerical algorithms with robust behavior.
In comparison to a standard spectral Galerkin discretizations we observe superior behavior with respect to relevant norms (i.e. the sup-norm for functions and the $L^{1}$-norm for densities)
across a range of resolutions.
We think these discretization will be useful for applications in optimal control and imaging when the user requires particularly low resolutions.

We apologize about the length and we hope that a length of 23 pages is acceptable.  We have strived to reduce the page length without sacrificing too much content.
Many nice theoretical results were discarded in this process.
We've found further reductions to be difficult to accomplish without sacrificing the main theorems of the paper or throwing out the numerical section entirely.

\closing{Sincerly,\\
%{\calligra \LARGE \color[rgb]{0,0,0.3} Henry O. Jacobs} \\
%\fromname{Dr. Henry O. Jacobs}
Henry O. Jacobs and Ram Vasudevan
}

\end{letter}
\end{document}