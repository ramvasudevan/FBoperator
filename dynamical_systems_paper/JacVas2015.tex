\documentclass[12pt]{amsart}
\usepackage{amsmath,amssymb}
\usepackage{geometry} % see geometry.pdf on how to lay out the page. There's lots.
\geometry{a4paper} % or letter or a5paper or ... etc
% \geometry{landscape} % rotated page geometry

%  POSSIBLY USEFULE PACKAGES
%\usepackage{graphicx}
%\usepackage{tensor}
\usepackage{todonotes}

%  NEW COMMANDS
\newcommand{\pder}[2]{\ensuremath{\frac{ \partial #1}{\partial #2}}}
\newcommand{\ppder}[3]{\ensuremath{\frac{\partial^2 #1}{\partial
      #2 \partial #3} } }
\newcommand{\R}{\ensuremath{\mathbb{R}}}

%  NEW THEOREM ENVIRONMENTS
\newtheorem{thm}{Theorem}[section]
\newtheorem{prop}[thm]{Proposition}
\newtheorem{cor}[thm]{Corollary}
\newtheorem{lem}[thm]{Lemma}
\newtheorem{defn}[thm]{Definition}


%  MATH OPERATORS
\DeclareMathOperator{\Diff}{Diff}
\DeclareMathOperator{\GL}{GL}
\DeclareMathOperator{\SO}{SO}
\DeclareMathOperator{\ad}{ad}
\DeclareMathOperator{\Ad}{Ad}

%  TITLE, AUTHOR, DATE
\title{An algebraically faithful and self-consistent discretization of advection}
\author{Henry O. Jacobs \& Ram Vasudevan}
\date{\today}


\begin{document}

\maketitle

\begin{abstract}
  blah blah blah.
\end{abstract}

\section{Introduction}
\label{sec:PDEs}
Let $X$ be a vector-field on a compact manifold $M$ of dimension $d$
and the evolution PDEs given in a local coordinate chart by
\begin{align}
	&\partial_t f + X^k \partial_k f= 0 \label{eq:function_pde} \\
	&\partial_t \rho + \partial_k( X^k \rho) = 0  \label{eq:density_pde} \\
	&\partial_t Y^i - (\partial_k X^i) \cdot Y^k + (\partial_k Y^i) X^k  = 0 \quad , \quad i = 1,\dots,d \label{eq:vf_pde}.
\end{align}
Each of these equations can be described as an advection equation in the sense that
\begin{itemize}
	\item a solution to \eqref{eq:function_pde} is given by a time-dependent function evolving under the dynamics generated by $X$,
	\item a solution to \eqref{eq:density_pde} is given by a time-dependent density evolving under the dynamics generated by $X$,
	\item and a solution to \eqref{eq:vf_pde} is given by a time-dependent vector-field evolving under the dynamics generated by $X$.
\end{itemize}
In summary, these PDEs can each be written in coordinate free notation as
\begin{align}
	\partial_t \alpha + \pounds_{X}[\alpha] = 0 \label{eq:advection_pde}
\end{align}
where $\alpha$ is a field and $\pounds_X[\alpha]$ is the Lie-derivative of $\alpha$ under the action of $X$.
The solution to \eqref{eq:advection_pde} (and therefore the solution to \eqref{eq:function_pde},\eqref{eq:density_pde}, and \eqref{eq:vf_pde}) is always of the form of a push-forward, $(\Phi_{X}^t)_* (\alpha_0)$, where $\Phi_{X}^t : M \to M$ is the flow at time $t \in \R$ induced by $X$ and $\alpha_0$ is the initial condition.

\subsection{Conservation properties of advection}
The advection of fields is a special type of evolution where the inter-relationships between the fields are conserved.
For example, if $f$ and $g$ are time-dependent functions which satisfy \eqref{eq:function_pde} then the functions $f+g$ and $fg$
obtained by point-wise addition and multiplication will satisfy \eqref{eq:function_pde} as well.
Similarly, if $f_t$ denotes the function $x \in M \to f(t,x) \in M$ then $\sup(f_t) = \sup(f_0)$ for any $t \in \mathbb{R}$.
In summary, the evolution induced by \eqref{eq:function_pde} is that of a time-dependent ring-morphism which is continuous with respect to 
the $C^0$-topology on $M$.
Of course, this implies that the constant function, $1$, is always a solution of \eqref{eq:function_pde}.

Another conserved structure is the relationship between functions and densities.  Given a function $f$ and a density $\rho$
one can form the real-number $\langle f, \rho \rangle$ obtained by integrating the density $f \rho$.
If $f$ and $\rho$ are solutions to \eqref{eq:function_pde} and \eqref{eq:density_pde} respectively, then $\langle f , \rho \rangle$ is constant in time.
In other words, advection preserves the dual-pairing between functions and densities\footnote{this can be extended to the space of distributions}.
It is notable that this property holds true for the constant function.
Thus advection yields a evolution of densities which is continuous with respect to the $L^1$-norm on densities. 

There is more.
If $Y_t$ is a solution to \eqref{eq:vf_pde} and $f_t$ is a solution to \eqref{eq:function_pde} then $\pounds_{Y_t}[f_t]$ is a solution to \eqref{eq:function_pde} too.
If $\rho_t$ is a solution to \eqref{eq:density_pde} then $\pounds_{Y_t}[ \rho_t]$ is a solution to \eqref{eq:density_pde} too.
If $Z_t$ is a solution to \eqref{eq:vf_pde} then $[Y_t,Z_t]$ is a solution to \eqref{eq:vf_pde} too.

In summary
\begin{itemize}
	\item Advection evolves the space of real-valued functions by ring-morphisms which preserve the $L^\infty$-norm.
	\item Advection evolves the space of densities (and distributions) by morphisms which preserve the dual pairing with 
		the functions, and the $L^1$-norm.
	\item Advection evolves the space of vector-fields by Lie-algebra morphisms.
\end{itemize}

\subsection{The purpose of this article}
\label{sec:purpose}
If one separately discretizes \eqref{eq:function_pde} \eqref{eq:density_pde} and \eqref{eq:vf_pde} using a legacy method
then, generally speaking, the resulting solutions will not satisfy the conservation properties just mentioned in any discretized sense.
For example, if $M$ is the $2$-torus and we discretize \eqref{eq:function_pde} with a finite difference scheme on a uniform grid,
then functions are finitely represented by their values at each grid point.
However, the evolution of the finite-difference scheme will not be that of a ring-morphism.
Moreover, if one uses the same finite-difference scheme to discretize \eqref{eq:density_pde}, the dual-pairing between the 
``discretized functions'' and the ``discretized densities'' will generally not be preserved.
The same holds for the other conservation properties mentioned.

Admittedly, it is a simple matter to work some of these properties into a discretization scheme.
For example, one could hypothetically discretized the linear operator  $\partial_k (X^k \cdot )$ appearing in \eqref{eq:density_pde} by the transpose of the
discretization of the linear operator $X^k \partial_k$ appearing in \eqref{eq:function_pde}.
This discretization would ensure that the dual-pairing between discrete functions and discrete densities is preserved.
However, given this new discretization,
it is not clear then how to incorporate other desired properties without breaking this dual-pairing preservation property.
In summary, it does not seem obvious how to incorporate all of these properties simultaneously.

The purpose of this article is to introduce a family of discretizations for generalized advection PDEs
which preserves all of the structures just mentioned at arbitrarily low resolutions.

\section{A reformulation of the continuous theory of advection}
\label{sec:continous}

Here is a basic outline of this section:
\begin{enumerate}
	\item Introduce $\mathfrak{X}(M)$ as space of derivations and relate this to the ring-morphism property.
	\item Introduce the Lie bracket of vector-fields
	\item Show how $\Diff(M)$ is the space of ring-automorphisms
	\item 
\end{enumerate}

\section{A discrete theory of advection}

\subsection{Topology}

\subsection{Algebra}

\section{Convergence}

\section{Applications}

\section{Examples}

\subsection{A one dimensional example}

\subsection{Duffing equation}

\subsection{Rigid body dynamics}
\todo[inline]{This is an integrable system.  It would be interesting to see the conserved functions (i.e. 0th Koopman modes)}

\subsection{A 3D example}
Lorenz equations?  Population dynamics?

\subsection{High dimensional example}
SRI models?  Quadrotor dynamics?  Chemical reaction dynamics?

\section{Conclusion}

\subsection{Open questions}
Koopmanism?  Noise?  Control?  Differential structures?

\subsection{Acknowledgements}

\bibliographystyle{amsalpha}
\bibliography{/Users/hoj201/Dropbox/hoj_2014.bib}
\end{document}
