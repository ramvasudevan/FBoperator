\documentclass[12pt]{amsart}
\usepackage{amsmath,amssymb}
\usepackage{geometry} % see geometry.pdf on how to lay out the page. There's lots.
\geometry{a4paper} % or letter or a5paper or ... etc
% \geometry{landscape} % rotated page geometry

%  POSSIBLY USEFULE PACKAGES
%\usepackage{graphicx}
%\usepackage{tensor}
\usepackage{todonotes}
\usepackage{xypic}

%  NEW COMMANDS
\newcommand{\pder}[2]{\ensuremath{\frac{ \partial #1}{\partial #2}}}
\newcommand{\ppder}[3]{\ensuremath{\frac{\partial^2 #1}{\partial
      #2 \partial #3} } }
\newcommand{\R}{\ensuremath{\mathbb{R}}}

%  NEW THEOREM ENVIRONMENTS
\newtheorem{thm}{Theorem}[section]
\newtheorem{prop}[thm]{Proposition}
\newtheorem{cor}[thm]{Corollary}
\newtheorem{lem}[thm]{Lemma}
\newtheorem{defn}[thm]{Definition}


%  MATH OPERATORS
\DeclareMathOperator{\Diff}{Diff}
\DeclareMathOperator{\Dens}{Dens}
\DeclareMathOperator{\SO}{SO}
\DeclareMathOperator{\Sym}{Sym}
\DeclareMathOperator{\Lin}{Lin}
\DeclareMathOperator{\Tr}{Tr}
\DeclareMathOperator{\Ad}{Ad}
\DeclareMathOperator{\Conj}{Conj}

%  TITLE, AUTHOR, DATE
\title{An algebraically faithful and self-consistent discretization of conservation laws}
\author{Henry O. Jacobs \& Ram Vasudevan}
\date{\today}


\begin{document}

\maketitle

\begin{abstract}
  blah blah blah.
\end{abstract}

\section{Introduction}
\label{sec:PDEs}
Let $X$ be a vector-field on a compact manifold $M$ of dimension $d$
and the evolution PDEs given in a local coordinate chart by
\begin{align}
	&\partial_t f + X^k \partial_k f= 0 \label{eq:function_pde} \\
	&\partial_t \rho + \partial_k( X^k \rho) = 0  \label{eq:density_pde} \\
	&\partial_t Y^i - (\partial_k X^i) \cdot Y^k + (\partial_k Y^i) X^k  = 0 \quad , \quad i = 1,\dots,d \label{eq:vf_pde}.
\end{align}
Each of these equations can be described as an advection equation in the sense that
\begin{itemize}
	\item a solution to \eqref{eq:function_pde} is given by a time-dependent function evolving under the dynamics generated by $X$,
	\item a solution to \eqref{eq:density_pde} is given by a time-dependent density evolving under the dynamics generated by $X$,
	\item and a solution to \eqref{eq:vf_pde} is given by a time-dependent vector-field evolving under the dynamics generated by $X$.
\end{itemize}
In summary, these PDEs can each be written in coordinate free notation as
\begin{align}
	\partial_t \alpha + \pounds_{X}[\alpha] = 0 \label{eq:advection_pde}
\end{align}
where $\alpha$ is a field and $\pounds_X[\alpha]$ is the Lie-derivative of $\alpha$ under the action of $X$.
The solution to \eqref{eq:advection_pde} (and therefore the solution to \eqref{eq:function_pde},\eqref{eq:density_pde}, and \eqref{eq:vf_pde}) is always of the form of a push-forward, $(\Phi_{X}^t)_* (\alpha_0)$, where $\Phi_{X}^t : M \to M$ is the flow at time $t \in \R$ induced by $X$ and $\alpha_0$ is the initial condition.

\subsection{Conservation properties of advection}
The advection of fields is a special type of evolution where the inter-relationships between the fields are conserved.
For example, if $f$ and $g$ are time-dependent functions which satisfy \eqref{eq:function_pde} then the functions $f+g$ and $fg$
obtained by point-wise addition and multiplication will satisfy \eqref{eq:function_pde} as well.
Similarly, if $f_t$ denotes the function $x \in M \to f(t,x) \in M$ then $\sup(f_t) = \sup(f_0)$ for any $t \in \mathbb{R}$.
In summary, the evolution induced by \eqref{eq:function_pde} is that of a time-dependent ring-morphism which is continuous with respect to 
the $C^0$-topology on $M$.
Of course, this implies that the constant function, $1$, is always a solution of \eqref{eq:function_pde}.

Another conserved structure is the relationship between functions and densities.  Given a function $f$ and a density $\rho$
one can form the real-number $\langle f, \rho \rangle$ obtained by integrating the density $f \rho$.
If $f$ and $\rho$ are solutions to \eqref{eq:function_pde} and \eqref{eq:density_pde} respectively, then $\langle f , \rho \rangle$ is constant in time.
In other words, advection preserves the dual-pairing between functions and densities\footnote{this can be extended to the space of distributions}.
It is notable that this property holds true for the constant function.
Thus advection yields a evolution of densities which is continuous with respect to the $L^1$-norm on densities. 

There is more.
If $Y_t$ is a solution to \eqref{eq:vf_pde} and $f_t$ is a solution to \eqref{eq:function_pde} then $\pounds_{Y_t}[f_t]$ is a solution to \eqref{eq:function_pde} too.
If $\rho_t$ is a solution to \eqref{eq:density_pde} then $\pounds_{Y_t}[ \rho_t]$ is a solution to \eqref{eq:density_pde} too.
If $Z_t$ is a solution to \eqref{eq:vf_pde} then $[Y_t,Z_t]$ is a solution to \eqref{eq:vf_pde} too.

In summary
\begin{itemize}
	\item Advection evolves the space of real-valued functions by ring-morphisms which preserve the $L^\infty$-norm.
	\item Advection evolves the space of densities (and distributions) by morphisms which preserve the dual pairing with 
		the functions, and the $L^1$-norm.
	\item Advection evolves the space of vector-fields by Lie-algebra morphisms.
\end{itemize}

\subsection{The purpose of this article}
\label{sec:purpose}
If one separately discretizes \eqref{eq:function_pde} \eqref{eq:density_pde} and \eqref{eq:vf_pde} using a legacy method
then, generally speaking, the resulting solutions will not satisfy the conservation properties just mentioned in any discretized sense.
For example, if $M$ is the $2$-torus and we discretize \eqref{eq:function_pde} with a finite difference scheme on a uniform grid,
then functions are finitely represented by their values at each grid point.
However, the evolution of the finite-difference scheme will not be that of a ring-morphism.
Moreover, if one uses the same finite-difference scheme to discretize \eqref{eq:density_pde}, the dual-pairing between the 
``discretized functions'' and the ``discretized densities'' will generally not be preserved.
The same holds for the other conservation properties mentioned.

Admittedly, it is a simple matter to work some of these properties into a discretization scheme.
For example, one could hypothetically discretized the linear operator  $\partial_k (X^k \cdot )$ appearing in \eqref{eq:density_pde} by the transpose of the
discretization of the linear operator $X^k \partial_k$ appearing in \eqref{eq:function_pde}.
This discretization would ensure that the dual-pairing between discrete functions and discrete densities is preserved.
However, given this new discretization,
it is not clear then how to incorporate other desired properties without breaking this dual-pairing preservation property.
In summary, it does not seem obvious how to incorporate all of these properties simultaneously.

The purpose of this article is to introduce a family of discretization for generalized advection PDEs
which preserves all of the structures just mentioned at arbitrarily low resolutions.

%%%%%%%%%%%%%%%%%%%%%%%%%%%%%%%%%%%%%%%%%%%%%%%%%%%
\section{The continuous regime}
\label{sec:continuous}
%%%%%%%%%%%%%%%%%%%%%%%%%%%%%%%%%%%%%%%%%%%%%%%%%%%
Let $M$ be closed compact smooth manifold.
By virtue of being a manifold, $M$ is canonically equipped with the following structures:
\begin{enumerate}
	\item The topological group of smooth diffeomorphisms, $(\Diff(M), \circ)$.
	\item The commutative ring of $C^k$ real-valued functions, $(C^k(M), + , \cdot)$.
	\item The Lie algebra of smooth vector-fields, $(\mathfrak{X}(M) , [ \cdot , \cdot ])$.
	\item The space of distributions, $\mathcal{D}'_k(M)$, dual to $C^k(M)$
	with the canonical pairing $\langle \cdot , \cdot \rangle : \mathcal{D}_k'(M) \times C^k(M) \to \R$.
	\item The cone of non-negative functions and the dual cone of non-negative densities.
	\item The inner product space of half-densities $(\Dens^{1/2}(M), \langle \cdot , \cdot \rangle)$.
\end{enumerate}
The last item may not be familiar to some readers and will be defined and elaborated on at the end of this section.
Each of these entities is linked to the others in a network of relationships
which we shall review here.
For starters, for any $\Phi \in \Diff(M)$ there exists an invertible operator, known as the \emph{Koopman operator}, given by
\begin{align*}
	f \in C^\infty(M) \mapsto \Phi^* f := f \circ \Phi \in C^\infty(M),
\end{align*}
where $f \circ \Phi$ denotes the composition of $f:M \to \R$ with $\Phi:M \to M$.
This is a (right) group action of $\Diff(M)$ on $C^k(M)$.
The Koopman operator satisfies
\begin{align*}
	\Phi^*(f+g) = \Phi^*f + \Phi^*g \text{ and } \Phi^*(f\cdot g) = (\Phi^*f) \cdot (\Phi^*g)
\end{align*}
for any two functions real-valued functions $f,g: M \to \mathbb{R}$.
In summary, the Koopman operator is a ring-automorphism of $C^k(M)$
and the diagram
\begin{equation}
	\xymatrix{
		C^0(M) \times C^0(M)  \ar@<-0.6ex>[d]_{+} \ar@<0.6ex>[d]^{\cdot} & \ar[l]_{\Phi^* \times \Phi^*} C^0(M) \times C^0(M)  \ar@<-0.6ex>[d]_{+} \ar@<0.6ex>[d]^{\cdot} \\
		C^0(M)  & \ar[l]_{\Phi^*} C^0(M)
	}
\end{equation}
commutes.

If we let $\Phi^t_X: M \to M$ be the time $t $ flow of some vector-field $X \in \mathfrak{X}$
we may define the Lie-derivative of a function $f$ to be
\begin{align*}
	\pounds_X[f] = \left. \frac{d}{d t} \right|_{t=0} (\Phi_X^t)^*f.
\end{align*}
In a local coordinate chart $(x^1,\dots,x^n)$ the vector-field is given by $X = X^i \pder{}{x^i}$
and the Lie-derivative is simply given by $\pounds_{X}[f] = X^i \pder{f}{x^i}$.
As the Koopman operator $(\Phi_X^t)^*$ is a ring-morphism, we find that the Lie-derivative operator is a derivation.
That is to say
\begin{align*}
	\pounds_X[ f \cdot g] = \pounds_X[f] \cdot g + f \cdot \pounds_X[g]
\end{align*}
for any $f,g \in C^k(M)$ with $k > 0$.

The topology of $C^0(M)$ is that of a Banach space with respect to the norm
\begin{align*}
	\| f \|_{C^0} = \sup_{x \in M} | f(x) | \equiv \| f \|_{L^\infty}
\end{align*}
We can observe that Koopman operator, $\Phi^*$, preserves this norm and is therefore
a continuous mapping of $C^0(M)$ to itself.
Therefore, we may form the continuous dual mapping $\Phi_*: \mathcal{D}_0'(M) \to \mathcal{D}_0'(M)$,
known by many as the \emph{Perron-Frobenius operator}.
Given a distribution, $\mu \in \mathcal{D}'_0(M)$, we define $\Phi_* \mu$ as the unique distribution such that
\begin{align*}
	\langle f , \Phi_*\mu \rangle = \int_M f \cdot \Phi_* \mu = \int_M (\Phi^*f ) \cdot \mu \quad \forall f \in C^0(M).
\end{align*}
As $\mathcal{D}'_0(M)$ is dual to $C^0(M)$, it must be topologized by the dual-norm.
In particular, $\mathcal{D}'_0(M)$ is a Banach space with respect to the norm
\begin{align*}
	\| \mu \|_{\mathcal{D}'_0} = \sup_{
		\substack{
				f \in C^0(M) \\
				\| f \|_{C^0} = 1
		}
	} \langle f , \mu \rangle =: \int_M | \mu | \equiv \| \mu \|_{L^1}
\end{align*}
The Perron-Frobenius operator preserves this norm.
These relationships are summarized by asserting that the diagram
\begin{equation*}
	\xymatrix{
		& \ar[dl]_{\| \cdot \|_{C^0}} C^0(M) & \ar[l]_{\pi_1} C^0(M)\times \mathcal{D}'_0(M) \ar[d]_{\langle \cdot , \cdot \rangle} \ar[r]^{\pi_2} & \mathcal{D}'_0(M)   \ar[dd]^{\Phi_*} \ar[dr]^{\| \cdot \|_{\mathcal{D}'}}& \\
		\mathbb{R} && \mathbb{R} && \mathbb{R} \\
		& \ar[ul]^{\| \cdot \|_{C^0}} C^0(M) \ar[uu]_{\Phi^*} & \ar[l]_{\pi_1} C^0(M)\times \mathcal{D}'_0(M) \ar[u]^{\langle \cdot , \cdot \rangle} \ar[r]^{\pi_2} & \mathcal{D}'_0(M) \ar[ur]_{\| \cdot \|_{\mathcal{D}'}} &	
	}
\end{equation*}
commutes for any $\Phi \in \Diff(M)$.

The cones of non-negative functions and non-negative distributions will play a special role later in terms of applications.
One can observe that the subring of non-negative functions, $C^k(M ; \R^+_0)$ is invariant under the Koopman operator.
That is to say, $\Phi^*f$ is a non-negative function if and only if $f$ is a non-negative function.
The same holds in regards to the Frobenius-Perron operator and non-negative distributions.

Finally, we introduce the inner-product space of half-densities.
A half-density is an object whose square is a density, and they act much like
the smooth wave-functions which one encounters in quantum mechanics.
More formally, a $C^k$ half density on $M$ is a $C^k$ multilinear function $\psi : \bigwedge^n TM \to \mathbb{R}$
such that for any matrix $A$ and vectors $v_1,\dots,v_n$ over the same fiber
\begin{align*}
	\psi( (A_1^j v_j) \wedge \cdots \wedge (A_n^j v_j) ) =  | \det(A) |^{1/2} \psi(v_1 \wedge \cdots \wedge v_n).
\end{align*}
We define the space of half-densities by $\Dens^{1/2}(M)$.
Such objects were first invented for the purpose of geometric quantization \cite[see Chapter 4]{GuilleminSternberg1970}.
Note that the function $\psi^2 : \bigwedge^n TM \to \R$ is a $C^k$ density, and can be integrated.
More generally, given $\psi_1,\psi_2 \in \Dens^{1/2}(M)$ we define $\langle \psi_1 \mid \psi_2 \rangle := \int_M \psi_1 \cdot \psi_2$
where $\psi_1 \cdot \psi_2 $ denotes the density obtained by scalar multiplication.
One can observe that $\langle \cdot \mid \cdot \rangle$ is an inner product on $\Dens^{1/2}(M)$.
Given a $\Phi \in \Diff(M)$, there is a canonical (left) action on $\Dens^{1/2}(M)$ given by an operator $\widehat{\Phi}$ defined by
\begin{align*}
	(\widehat{\Phi} \cdot \psi)( v_1 \wedge \cdot \wedge v_n) := \psi( D\Phi^{-1}(v_1) \wedge \cdots \wedge D\Phi^{-1}(v_n) )
\end{align*}
where $D\Phi^{-1} : TM \to TM$ denotes the tangent lift of $\Phi^{-1}:M \to M$.
By the change of variables formula we see that $\Diff(M)$ acts isometrically on $\Dens^{1/2}(M)$ with respect to 
the inner product $\langle \cdot \mid \cdot \rangle$.
The conservation properties of $\Phi$ with regards to half-densities can be summarized by asserting that the diagram
\begin{equation*}
	\xymatrix{
		\Dens^{1/2}(M) \times \Dens^{1/2}(M) \ar[dr]_{\langle \cdot \mid \cdot \rangle} \ar[r]^{\widehat{\Phi} \times \widehat{\Phi}} & \Dens^{1/2}(M) \times \Dens^{1/2}(M) \ar[d]^{\langle \cdot \mid \cdot \rangle} \\
		&\mathbb{R}
	}
\end{equation*}
commutes.

As $\Diff(M)$ acts isometrically upon $\Dens^{1/2}(M)$ the Lie algebra $\mathfrak{X}(M)$ must act by anti-symmetric operators.
In particular, we define the Lie derivative of a half-density $\psi \in \Dens^{1/2}(M)$ by
\begin{align*}
	\pounds_X[ \psi ] = \frac{d}{dt} |_{t=0} \widehat{\Phi}_t^T \psi
\end{align*}
where $\Phi_t$ is the time $t \in \R$ flow of $X$.
In coordinates the Lie derivative is given by
\begin{align*}
	\pounds_X[\psi] = \frac{1}{2} X^i \pder{\psi}{x^i} + \frac{1}{2} \pder{}{x^i} \left(\psi \cdot X^i \right)
\end{align*}
Admittedly, half-densities are a somewhat obscure object which would be of no interest to use, except for the fact that they will guide
our discretization procedure in section \ref{sec:convergence}.

%%%%%%%%%%%%%%%%%%%%%%%%%%%%%%%%%%%%%%%%%%%%%%%%%%%
\section{The discrete regime}
\label{sec:discrete}
%%%%%%%%%%%%%%%%%%%%%%%%%%%%%%%%%%%%%%%%%%%%%%%%%%%
\todo[inline]{Consider replacing $\R^n$ with a generic Hilbert space $\mathcal{H}$ and retitling the section as operator theory form or something.
This makes the next section an easier transition.}
In this section we review discrete analogs of the structures provided in the section \ref{sec:continuous}.
In particular, this section is literally a copy of the previous section, except that all the infinite-dimensional entities have been replaced with finite dimensional ones.
Consider the space $\R^n$ equipped with the following canonical structures:
\begin{enumerate}
	\item The Lie group of isometries, $\SO(n)$, with lie algebra $\mathfrak{so}(n)$.
	\item The noncommutative ring of linear operators $\Lin(n)$.
	\item The dual-space $\Lin(\R^n)^* \cong \Lin(\R^n)$.
	with the canonical pairing $\langle A , \mu \rangle = \Tr(A \mu)$.
	\item The cone of positive-definite symmetric matrices $\Sym(n)^+$ and the dual cone $(\Sym(n)^+)^*$.
\end{enumerate}
Each of these entities is linked to the others in a network of relationships
which we shall review here.
These relationships will suggest the following analogies
\begin{itemize}
	\item The group $\SO(n)$ is analogous to $\Diff(M)$
	\item The Lie algebra $\mathfrak{so}(n)$ is analogous to the Lie algebra $\mathfrak{X}(M)$.
	\item The noncommutative ring $\Lin(n)$ is analogous to the commutative ring $C^0(M)$.
	\item The dual-space $\Lin(n)^*$ is analogous to $\mathcal{D}'_0(M)$.
	\item The cone of positive semi-definite matrices $\Sym_0^+(n)$ is analogous to the space of non-negative functions on $M$.
	\item The inner-product space $\R^n$ is analogous to the inner-product space $\Dens^{1/2}(M)$.
\end{itemize}
For starters, for any $R \in \SO(n)$ there exists a \emph{discrete Koopman operator}, $R^*$, given by
\begin{align*}
	\hat{f} \in \Lin(n) \mapsto R^* ( \hat{f} ) := R^T \hat{f} R \in \Lin(n)
\end{align*}
This is a (right) group action of $\SO(n)$ on $\Lin(n)$.
The discrete Koopman operator satisfies
\begin{align*}
	R^*(\hat{f}+\hat{g}) = R^*(\hat{f}) + R^*(\hat{g}) \intertext{ and }
	R^*(\hat{f} \cdot \hat{g}) = R^*(\hat{f}) \cdot R^*(\hat{g})
\end{align*}
for any two $\hat{f},\hat{g} \in \Lin(n)$.
In summary, the discrete Koopman operator is a ring-automorphism of $\Lin(n)$.
This is summarized by noting the diagram
\begin{equation}
	\xymatrix{
		\Lin(n) \times \Lin(n)  \ar@<-0.6ex>[d]_{+} \ar@<0.6ex>[d]^{\cdot} & \ar[l]_{R^*} \Lin(n) \times \Lin(n)  \ar@<-0.6ex>[d]_{+} \ar@<0.6ex>[d]^{\cdot} \\
		\Lin(n)  & \ar[l]_{R^*} \Lin(n)
	} \label{cd:discrete_functions}
\end{equation}
commutes.

If we let $R_t = \exp( \omega t)$ for some $\omega \in \mathfrak{so}(n)$
we may define the \emph{discrete Lie-derivative} of $\hat{f} \in \Lin(n)$ to be
\begin{align*}
	\omega[ \hat{f}] := \left. \frac{d}{d t} \right|_{t=0} R_t^* (\hat{f} ) = [ \hat{f} , \omega].
\end{align*}
where $[ \cdot , \cdot ]$ denotes the matrix commutator.
As the discrete Koopman operator $\Conj_{R_t}$ is a ring-morphism, we find that the discrete Lie-derivative operator is a (non-commutative) derivation.
That is to say
\begin{align*}
	\omega[ \hat{f} \cdot \hat{g} ] = \omega[ \hat{f}] \cdot g + f \cdot \omega[\hat{g}]
\end{align*}
for any $\hat{f},\hat{g} \in \Lin(n)$.

The space $\Lin(n)$ is naturally equipped with the operator norm
\begin{align*}
	\| \hat{f} \|_{\Lin} = \sup_{x \in \R^n} \| \hat{f} \cdot x \|_{\R^n}.
\end{align*}
where $\| \cdot \|_{\R^n}$ denotes the $2$-norm on $\R^n$.
We can observe that discrete Koopman operator preserves this norm.

We may also form the dual mapping $R_*: \Lin(n)^* \to \Lin(n)^*$,
which we shall dub the \emph{discrete Perron-Frobenius operator}.
Given a $\mu \in \Lin(n)^*$, we may define $R_*(\mu)$ by the constraint
\begin{align*}
	\langle \hat{f} , R_*(\mu) \rangle = \langle R^*(\hat{f}) , \mu \rangle \quad \forall \hat{f} \in \Lin(n).
\end{align*}
This implies $R_*(\mu) = R \mu R^T$.
We naturally equip $\Lin(n)^*$ with the Banach norm dual to that chosen on $\Lin(n)$.
In particular, $\Lin(n)$ is  equipped with the norm
\begin{align*}
	\| \mu \|_{\Lin^*} = \sum_{s \in \Sigma(\mu) } s
\end{align*}
where $\Sigma(\mu)$ denotes the set of singular values of $\mu$.
The discrete Perron-Frobenius operator preserves this norm.
These relationships are summarized in the commutativity of the diagram
\begin{equation*}
	\xymatrix{
		& \ar[dl]_{\| \cdot \|_{\Lin}} \Lin(n) & \ar[l]_{\pi_1} \Lin(n)\times \Lin(n)^* \ar[d]_{\langle \cdot , \cdot \rangle} \ar[r]^{\pi_2} & \Lin(n)^*   \ar[dd]^{R_*} \ar[dr]^{\| \cdot \|_{\Lin^*}}& \\
		\mathbb{R} && \mathbb{R} && \mathbb{R} \\
		& \ar[ul]^{\| \cdot \|_{\Lin}} \Lin(n) \ar[uu]_{R^*} & \ar[l]_{\pi_1} \Lin(n)\times \Lin(n)^* \ar[u]^{\langle \cdot , \cdot \rangle} \ar[r]^{\pi_2} & \Lin(n)^* \ar[ur]_{\| \cdot \|_{\Lin^*}} &	
	}
\end{equation*}

The cones of positive semi-definite matrices and the dual cone will play a special role.
One can observe that the partial subring $\Sym(n)^+$ is invariant under the Koopman operator.
That is to say, $R^*(\hat{f}) \in \Sym(n)^+$ if and only if $\hat{f} \in \Sym(n)^+$.
The same holds in regards to the discrete Frobenius-Perron operator on $(\Sym(n)^+)^*$.

Finally, we note that the very definition of $\SO(n)$ is that its elements preserve the inner-product on $\R^n$.
In other words the diagram
\begin{equation*}
	\xymatrix{
		\R^n \times \R^n \ar[dr]_{\langle \cdot \mid \cdot \rangle} \ar[r]^{R^*} & \R^n \times \R^n \ar[d]^{\langle \cdot \mid \cdot \rangle} \\
		&\mathbb{R}
	}
\end{equation*}
commutes.

At this point a reader may come to the conclusion that this is all very cute, but
what good is this analogy unless we provide a link between the discrete and the continuous regimes?
This is the topic of the next section.

%Finally, we introduce the inner-product space of half-densities.
%Roughly speaking, a half-density is an object whose square is a density, and they act much like
%the smooth wave-functions which one encounters in quantum mechanics.
%More formally, a smooth half density on $M$ is a smooth multiliear function $\psi : \bigwedge^n TM \to \mathbb{R}$
%such that for any matrix $A$ and vectors $v_1,\dots,v_n$ over the same fiber
%\begin{align*}
%	\psi( (A_1^j v_j) \wedge \cdots \wedge (A_n^j v_j) ) =  | \det(A) |^{1/2} \psi(v_1 \wedge \cdots \wedge v_n).
%\end{align*}
%We define the space of half-densities by $\Dens^{1/2}(M)$.
%Such objects were first invented for the purpose of geometric quantization \cite[see Chapter 4]{GuilleminSternberg1970}.
%Note that the function $\psi^2 : \bigwedge^n TM \to \R$ is a density, and can be integrated.
%More generally, given $\psi_1,\psi_2 \in \Dens^{1/2}(M)$ we define $\langle \psi_1 \mid \psi_2 \rangle := \int_M \psi_1 \cdot \psi_2$
%where $\psi_1 \cdot \psi_2 $ denotes the density obtained by scalar multiplication.
%One can observe that $\langle \cdot \mid \cdot \rangle$ is an inner product on $\Dens^{1/2}(M)$.
%Given a $\Phi \in \Diff(M)$, there is a canonical (right) action on $\Dens^{1/2}(M)$ given by
%\begin{align*}
%	\Phi^* \psi( v_1 \wedge \cdot \wedge v_n) := \psi( D\Phi(v_1) \wedge \cdots \wedge D\Phi(v_n) )
%\end{align*}
%where $D\Phi : TM \to TM$ denotes the tangent lift of $\Phi$.
%By the change of variables formula we see that $\Diff(M)$ acts isometrically on $\Dens^{1/2}(M)$ with respect to 
%the inner product $\langle \cdot \mid \cdot \rangle$.
%Admittedly, this is an obscure object which would be of no interest to use, except for the fact that it will guide
%our discretization procedure in the next section.
%
%We can summarize what has been presented so far in the following commutative diagrams
%\begin{align*}
%	\xymatrix{
%		\Lin(n) \times \Lin(n)  \ar@<-0.6ex>[d]_{+} \ar@<0.6ex>[d]^{\cdot} & \ar[l]_{R^*} \Lin(n) \times \Lin(n)  \ar@<-0.6ex>[d]_{+} \ar@<0.6ex>[d]^{\cdot} \\
%		\Lin(n)  & \ar[l]_{R^*} \Lin(n)
%	}, \\
%	\xymatrix{
%		& \ar[dl]_{\| \cdot \|_{\Lin}} \Lin(n) & \ar[l]_{\pi_1} \Lin(n)\times \Lin(n)^* \ar[d]_{\langle \cdot , \cdot \rangle} \ar[r]^{\pi_2} & \Lin(n)^*   \ar[dd]^{(R^*)^*} \ar[dr]^{\| \cdot \|_{\Lin^*}}& \\
%		\mathbb{R} && \mathbb{R} && \mathbb{R} \\
%		& \ar[ul]^{\| \cdot \|_{\Lin}} \Lin(n) \ar[uu]_{R^*} & \ar[l]_{\pi_1} \Lin(n)\times \Lin(n)^* \ar[u]^{\langle \cdot , \cdot \rangle} \ar[r]^{\pi_2} & \Lin(n)^* \ar[ur]_{\| \cdot \|_{\Lin^*}} &	
%	} \\
%	\xymatrix{
%		\R^n \times \R^n \ar[dr]_{\langle \cdot \mid \cdot \rangle} \ar[r]^{R^*} & \R^n \times \R^n \ar[d]^{\langle \cdot \mid \cdot \rangle} \\
%		&\mathbb{R}
%	}	.
%\end{align*}


\section{Convergence}
\label{sec:convergence}
Let $M$ be a compact manifold.
The inner-product space, $\Dens^{1/2}(M)$, can be completed to yield a separable Hilbert space, $L^2(M)$,
and we may choose a countable Hilbert basis $\{e_0,e_1,\dots\}$.\footnote{This definition of $L^2(M)$ may appear strange but it is identical to the one usually introduced with one small difference.
This definition does not require one to choose a reference density in order to create integrals of functions.}

In order to link section \ref{sec:continuous} and section \ref{sec:discrete}
we first need to reformulate our representations of functions, densities, and diffeomorphisms as operators on $L^2(M)$.
To begin, for each function $f \in C^\infty(M)$ there is a unique symmetric linear operator $\hat{f} \in \Lin( L^2(M) )$
obtained by point-wise multiplication.  Explicitly, $(\hat{f} \cdot \psi)(x) := f(x) \psi(x)$.
The mapping $f \mapsto \hat{f}$ is a ring-monomorphism in the non-commutative ring of operators on $L^2(M)$.
If we let $\| \cdot \|_{Op}$ denote the operator norm, then $\| \hat{f} \|_{Op} = \| f \|_{\infty}$.
%Note that $\Lin(L^2(M))$ is an inner-product space with the inner-product $\langle A \mid B \rangle = \Tr( A^T B)$.
By duality, for each density $\rho \in \mathcal{D}'(M)$ there is a unique symmetric operator $\hat{\rho} \in \Lin(L^2(M))^* \cong \Lin(L^2(M))$ defined
such that $ \hat{\rho} (\hat{f}) = \int_M f \cdot \rho $.
If we let $\| \cdot \|_{Op^*}$ denote the dual-norm to the operator norm, then $\| \hat{\rho} \|_{Op^*} = \| \rho \|_{L^1}$.
Finally, given a $\Phi \in \Diff(M)$ one can then verify
\begin{align*}
	\widehat{\Phi^* f} = \widehat{\Phi}^T \cdot \hat{f} \cdot \widehat{\Phi} \quad \text{ and } \quad
	\widehat{\Phi_* \rho} = \widehat{\Phi} \cdot \hat{\rho} \cdot \widehat{\Phi}^T
\end{align*}
for any $f \in C^0(M)$ and $\rho \in \mathcal{D}'(M)$.
These ``hat''-maps make the structures of section \ref{sec:continuous} identical to the structures of section \ref{sec:discrete}
except $\R^n$ is replaced with $L^2(M)$.

\section{Applications}

\subsection{A one dimensional example}

\subsection{Duffing equation}

\subsection{Rigid body dynamics}
\todo[inline]{This is an integrable system.  It would be interesting to see the conserved functions (i.e. 0th Koopman modes)}

\subsection{A 3D example}
Lorenz equations?  Population dynamics?

\subsection{High dimensional example}
SRI models?  Quadrotor dynamics?  Chemical reaction dynamics?

\section{Conclusion}

\subsection{Open questions}
Koopmanism?  Noise?  Control?  Differential structures (e.g. the exterior derivative)?  The link with non-commutative geometry.

\subsection{Acknowledgements}

\bibliographystyle{amsalpha}
\bibliography{hoj.bib}
\end{document}
