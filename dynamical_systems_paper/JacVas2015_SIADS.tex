\documentclass[12pt]{amsart}
\usepackage{amsmath,amssymb}
\usepackage{geometry} % see geometry.pdf on how to lay out the page. There's lots.
\geometry{a4paper} % or letter or a5paper or ... etc
% \geometry{landscape} % rotated page geometry

%  POSSIBLY USEFULE PACKAGES
%\usepackage{graphicx}
\usepackage{tensor}
\usepackage{todonotes}
\usepackage{xypic}

%  NEW COMMANDS
\newcommand{\pder}[2]{\ensuremath{\frac{ \partial #1}{\partial #2}}}
\newcommand{\ppder}[3]{\ensuremath{\frac{\partial^2 #1}{\partial
      #2 \partial #3} } }
\newcommand{\R}{\ensuremath{\mathbb{R}}}
\renewcommand{\H}{\ensuremath{\mathcal{H}}}
\newcommand{\norm}[1]{\ensuremath{\left\| #1 \right\| }}
\newcommand{\abs}[1]{\ensuremath{ | #1 | }}

%  NEW THEOREM ENVIRONMENTS
\newtheorem{thm}{Theorem}[section]
\newtheorem{prop}[thm]{Proposition}
\newtheorem{cor}[thm]{Corollary}
\newtheorem{lem}[thm]{Lemma}
\newtheorem{defn}[thm]{Definition}

% NEW ENVIRONMENTS
\newenvironment{example}
  { \newline {\bf Example:} }
  {\rule{1ex}{1ex} }

%  MATH OPERATORS
\DeclareMathOperator{\Diff}{Diff}
\DeclareMathOperator{\Dens}{Dens}
\DeclareMathOperator{\SO}{SO}
\DeclareMathOperator{\U}{U}
\DeclareMathOperator{\B}{B}
\DeclareMathOperator{\Fr}{Fr}
\DeclareMathOperator{\Herm}{Herm}
\DeclareMathOperator{\Op}{Op}
\DeclareMathOperator{\Tr}{Tr}
\DeclareMathOperator{\card}{card}
\DeclareMathOperator{\Con}{Con}

%  TITLE, AUTHOR, DATE
\title{Qualitatively accurate spectral schemes for advection}
\author{Henry O. Jacobs \& Ram Vasudevan}
\date{\today}


\begin{document}

\maketitle

\begin{abstract}
  blah blah blah.
\end{abstract}

%%%%%%%%%%%%%%%%%%%%%%%%%%%%%%%%%%%%%%%%%%%%%%%%%%%
\section{Introduction}
\label{sec:intro}

Qualitative accuracy is important because...
Here is an example of a failure.

\subsection{Previous work}

%%%%%%%%%%%%%%%%%%%%%%%%%%%%%%%%%%%%%%%%%%%%%%%%%%%
\section{Derivation and properties of advection PDEs}
\label{sec:properties}
In this section we define specifically what we mean by ``advection PDEs'' in more precise mathematical terms.
Throughout this paper let $\mathfrak{X}(M)$ denote the space of smooth vector-fields on a compact manifold $M$
and let $X(t) \in \mathfrak{X}(M)$ be a time-dependent vector-field on $M$.
Let $\Phi^{t_{0},t_{1}}_{X}:M\to M$ denote the flow map of $X$ from time $t_{0}$ to $t_{1}$.
Given a bounded real-valued function $f_{0}$, we could consider the time-dependent function $f(x,t) := f_{0} ( (\Phi_{X}^{0,t})^{-1}(x) )$.
The function $f$ represents how $f_{0}$ is pushed and stirred around by the flow of $X$.
For a fixed $x \in M$ we observe, via the chain-rule,
\begin{align*}
	 \frac{d}{dt}  f(x;t) &=  \left. \frac{d}{d \epsilon} \right|_{\epsilon=0} ( f(x;t+\epsilon) - f(x,t) ) \\
	 &=  \left. \frac{d}{d \epsilon} \right|_{\epsilon=0} \left( f( (\Phi_{X}^{t,t+\epsilon})^{-1}(x) , t ) - f(x,t) \right) \\
	 &=  \left. \frac{d}{d \epsilon} \right|_{\epsilon=0} \left( f(x,t) + \epsilon \pder{f}{x^{j}} [ (\Phi_{X}^{t,t+\epsilon})^{-1}(x) - x ]^{j} + \mathcal{O}(\epsilon^{2}) - f(x,t) \right) \\
	 &= - X^{j}(x) \pder{f}{x^{j}} = - \pounds_{X}[f_{0}]
\end{align*}
where $\pounds_{X}[f]$ denotes the Lie-derivative of $f$.
This calculation yields the time-dependent advection PDE
\begin{align}\label{eq:function pde}
	\pder{f}{t} + X^{j} \pder{f}{x^{j}} = 0
\end{align}

The same idea works if we consider the advection of a density $\rho_{0}$ by $X$.
In this case, the advected density is given by $\rho (x ; t) =  \det\left[ (D\Phi_{X}^{t,0})^{-1} (x) \right] \rho_{0}( (\Phi_{X}^{0,t})^{-1}(x) )$
and the advection pde is
\begin{align} \label{eq:density pde}
	\pder{\rho}{t} + \pder{}{x^{j}} \left( \rho X^{j} \right) = 0.
\end{align}

Similarly, a (constant in time) vector-field $Y_{0}$ is advected by $X$ by the formula $Y(x ; t) =  (D\Phi_{X}^{t,0})|_{(\Phi_{X}^{t,0})^{-1}(x)} \cdot Y_{0}( (\Phi_{X}^{t,0})^{-1}(x))$
and satisfies the advection PDE
\begin{align} \label{eq:vector field pde}
	\pder{Y^{i}}{t} + \pder{Y^{i}}{x^{j}} \cdot X^{j} - \pder{X^{i}}{x^{j}} \cdot Y^{j} = 0.
\end{align}

\section{Qualitative properties of advection}
\label{sec:properties}
Advection PDEs preserve a lot of natural structures.
For example if $f(x,t)$ and $g(x,t)$ are solutions to \eqref{eq:function pde}, then the scalar product $f(x,t) g(x,t)$  and the sum $f(x,t) + g(x,t)$ is also a solution to \eqref{eq:function pde}.
This observation can be summarized by the statement \emph{evolution by \eqref{eq:function pde} preserves the ring of functions on $M$}.
In the same vein, if $\rho$ is a density which satisfies \eqref{eq:density pde} and $f$ is a function which satisfies \eqref{eq:function pde}, then the quantity $c = \int_{M} f \cdot \rho$ is constant in time.
This can naturally be extended to the case where $\rho$ is a distribution, and we can summarize this conservation law by saying \emph{advection preserves the duality between functions and distributions on $M$}.

More conservation laws can be found for vector-field advection.
If $Y$ and $Z$ are vector-fields, then there is a natural Lie-bracket structure given by
\begin{align*}
	[Y,Z]^{i} \equiv ( \pounds_{Y}[Z] )^{i} := \partial_{j}Z^{i} \, Y^{j} - \partial_{j}Y^{i} \, Z^{j}.
\end{align*}
We can observe that if $Y$ and $Z$ satisfy \eqref{eq:vector field pde} then $[Y,Z]$ also satisfies \eqref{eq:vector field pde}.
In summary, advection preserves the Lie algebra of vector fields.

Lastly, the canonical relationships between vector-fields and functions and densities are also preserved.
Given a vector-field $Y$ a function $f$ and a density $\rho$ we may define the Lie derivatives
\begin{align*}
	\pounds_{Y}[f] = Y^{j} \partial_{j}f \quad , \quad \pounds_{Y}[\rho] = \pder{}{x^{j}} ( \rho Y^{j} ).
\end{align*}
The quantity $\pounds_{Y}[f]$ is itself a function, and $\pounds_{Y}[\rho]$ is a density.
If $f,\rho$ and $Y$ satisfy the advection PDEs \eqref{eq:function pde}, \eqref{eq:density pde}, and \eqref{eq:vector field pde} respectively,
then $\pounds_{Y}[f]$ satisfies \eqref{eq:function pde} and $\pounds_{Y}[\rho]$ satisfies \eqref{eq:density pde}.
In summary, advection preserves the action of vector fields on functions and densities.

In some sense these conservation laws are remarkable, since most PDEs do not conserve all the structures mentioned.
Of course, there is no such thing as magic.
The group of smooth diffeomorphisms of $M$ is in fact identical to the group of ring-automorphisms of $C^{\infty}(M)$.
The formal Lie algebra of the group of Diffeomorphisms is the space of vector-fields $\mathfrak{X}(M)$.
So it is no wonder that advection preserves the ring-structure of functions.
Moreover, the space of distributions is \emph{defined} as the dual space to that of functions, and so the action of a vector-field on distributions is defined
as the dual of the action on functions.
So of course advection preserves the duality between functions and distributions.
Lastly, any Lie group preserves it's own Lie algebra and its own group actions.
So of course advection preserves the Lie bracket of vector fields and the Lie derivatives of functions and densities.
These conservation laws are fundamental properties of advection, and this motivates the definition of qualitative accuracy
which we shall use in this paper:

\fbox{
	\begin{minipage}{0.8\textwidth}
		A numerical method for an advection is {\bf qualitatively accurate} if it conserves (up to machine precision) the same properties as the exact advection pde.
	\end{minipage}
}

This criterion is separate from numerical accuracy, and can be evaluated independently.
Of course, numerical accuracy is a first and priority,
but the goal of this paper will be to go the extra length of constructing a scheme which is numerically
and qualitatively accurate.

%%%%%%%%%%%%%%%%%%%%%%%%%%%%%%%%%%%%%%%%%%%%%%%%%%%
\section{The canonical space $L^{2}(M)$}
\label{sec:half densities}
The goal of this paper is to construct a spectral advection scheme which exhibits a large number of invariants.
At the core of many spectral schemes is the use of some Hilbert space upon which everything can be approximated via least squares projections.
Typically this Hilbert space is $L^{2}(M ; \mu)$ where the $L^{2}$-inner product is taken with respect to some non-canonical density or volume form $\mu$ on $M$.
We do not have this luxury.
The properties of advection that we desire to have are strictly properties of $M$ alone, i.e. they are canonical.
Any non-canonical choices we make in constructing our scheme (such as the choice of a volume form) threaten to break these conservation laws by introducing non-canonical artifacts.

In this section we define a geometric version of an $L^{2}$-space which foregoes the choice of a volume form, denoted simply by $L^{2}(M)$.
The ``cost'' we pay for not choosing a volume form is that the elements of $L^{2}(M)$ are not (equivalence classes of) functions because they transform differently under a change of coordinates.

Again, let $M$ be a smooth compact $n$-manifold
and we will let $\Dens(M)$ denote the space of continuous densities,
which we view as anti-symmetric multilinear functions on $\bigoplus^n TM$ \cite[Chapter 16]{Lee2006}.

\begin{defn}\label{def:half density}
	A half-density is a continous complex function $\psi : \bigoplus^n TM \to \mathbb{C}$
	such that $| \psi |^{2} \in \Dens(M)$.
	We denote the space of half densities by $\sqrt{\Dens(M)}$.
\end{defn}

This definition is an equivalent reformulation of the half densities defined in the context of geometric quantization \cite{GuilleminSternberg1970,BatesWeinstein1997}.
In physical terms, $L^{2}(M)$ is precisely the Hilbert space of wave functions on $M$ used in quantum mechanics.
It is unfortunate that physicists call these ``wave-functions''
because elements of $L^{2}(M)$ and $\sqrt{\Dens(M)}$ are \emph{not} functions on $M$.\footnote{However, this sloppy use of names is consistent with others instances such as the fact that the ``Dirac-delta function'' is also not a function.}
One way to see this is to observe how elements of $\sqrt{\Dens(M)}$ transform.
Under a change of coordinates $\Phi$ we observe $\psi$ will map to $\tilde \psi$ where
\begin{align}
	\tilde{\psi}(x)  =  \left| \det \left[ \left. \pder{\Phi^{i}}{x^{j}} \right|_{\Phi^{-1}(x)} \right]_{i,j=1}^{n} \right|^{1/2} \psi( \Phi^{-1}(x) ). \label{eq:transformation law}
\end{align}


As $|\psi|^{2} \in \Dens(M)$ for any $\psi \in \sqrt{\Dens(M)}$ we can integrate $|\psi|^{2}$
and observe that half densities are is naturally equipped with the norm $$\| \nu \|_2 :=  \left( \int_M |\nu|^2 \right)^{1/2}$$ which we call the $2$-norm.

\begin{defn}
	We define $L^{2}(M)$ as the completion of $\sqrt{ \Dens(M)}$ with respect to the $2$-norm.
	The space $L^{2}(M)$ is equipped with an inner-product given by
	\begin{align*}
		\langle \psi \mid \phi \rangle = \int_{M} \bar \psi \phi
	\end{align*}
	through polar decomposition, and so $L^{2}(M)$ is a Hilbert space.
\end{defn}




\subsection{The relationship with classical $L^{2}$ spaces}
\label{sec:classical_Lebesgue}
For any manifold $M$ (possibly non-orientable) one can assert the existence of a smooth non-negative density $\mu$.
In any case, upon choosing some $\mu$ the $2$-norm of a complex function $f$ with respect to $\mu$
\begin{align*}
	\| f \|_{\mu,2} =  \left( \int_M |f|^2 \mu \right)^{1/2}.
\end{align*}
and $L^2(M ; \mu)$ is the completion of the space of continuous functions with respect to this norm.

\begin{prop}
	Choose a non-vanishing density $\mu$.
	Let $\sqrt{\mu}$ be any half-density such that $| \sqrt{\mu} |^{2} = \mu$.
	For any $\psi \in L^2(M)$ there exists a unique $f \in L^2(M ; \mu)$ such that $\psi = f \cdot  \sqrt{\mu}$.
	This yields a continuous isomorphism between $L^2(M)$ and $L^2(M ; \mu)$.
\end{prop}
\begin{proof}
	First note that if we view $\mu$ as a positive function on $\bigoplus^n TM$, then we can simply set $\sqrt{\mu}$ to be it's classical square root.
	It suffices to prove that the dense subspace $\sqrt{\Dens(M)} \subset L^{2}(M)$ can be mapped to a dense subspace of $L^{2}(M;\mu)$ through a map that sends $\| \cdot \|_{2}$ to $\| \cdot \|_{\mu,2}$.
	
	Let $\psi \in \sqrt{\Dens(M)}$.  Then $|\psi|^2$ is a continuous density.
	The Radon-Nikodym derivative of $|\psi|^{2}$ with respect to $\mu$ is a positive valued continuous function $g \in C^0(M)$ defined such that $g \, \mu = |\psi|^2$.
	Without loss of generality, we may set $g = |f|^{2}$ for some complex function $f$ such that $\psi = f\, \sqrt{\mu}$.
	The function $f$ is unique with respect to $\psi$ and
	the map $\psi \in \sqrt{\Dens(M)} \mapsto f \in C^0(M ; \mathbb{C} )$ sends $\| \cdot \|_{2}$ to $\| \cdot \|_{\mu,2}$ by construction.
	Thus the map is continuous.
	The inverse of the map is given by $f \in C^{0}(M;\mathbb{C}) \mapsto f \, \sqrt{\mu} \in \sqrt{\Dens(M)}$.
\end{proof}

\subsection{Sobolev spaces on Riemannian manifolds}
\label{sec:Sobolev spaces}
Let $M$ be a Riemannian $n$-manifold with metric $g:TM \oplus TM \to \mathbb{R}$.
The metric induces a positive density $\mu_g$,
known as the \emph{metric density},
as well as an elliptic operator,
$\Delta:C^\infty(M) \to C^{\infty}(M)$,
known as the \emph{Laplace-Beltrami operator}. 
The Riemmanian density induces the $L^2$ inner-product on $C^\infty(M)$ via
\begin{align*}
	\langle f_1 , f_1 \rangle_{g} = \int \overline{f_1} \cdot f_2 \mu_g
\end{align*}
and $\Delta$ is positive-semidefinite.
If $M$ is compact, then $L^2(M ; \mu_g) \cong L^2(M)$ is a separable Hilbert space
and the Helmholtz operator, $1 + \Delta$, is then a positive definite operator
with a discrete spectrum.
For any $s \geq 0$ we may define the Sobolev norm
\begin{align*}
	\| f \|_{s,2} =  \left( \langle f , (1+\Delta)^s \cdot f \rangle_{g} \right)^{1/2}
\end{align*}
and $H^s(M ; g)$ is then the function space obtained by completion with respect to this norm, and modulo discrepancies of measure zero.
Note that $H^0(M;g) = L^2(M;\mu_g)$.

The following proposition will be of use later when we need to prove a notion of convergence
via sequences of finite rank operators.

\begin{prop} \label{prop:compact_embedding}
	Let $(M,g)$ be a compact Riemmanian manifold.  If $s > t \geq 0$ then $H^s(M;g)$ is compactly embedded within $H^t(M,g)$.
\end{prop}
\begin{proof}
	Let $e_0, e_1,\dots$ be a Hilbert basis for $L^2(M;\mu_g)$ which diagonalizes $\Delta$
	in the sense that $\Delta e_i = \lambda_i e_i$ for a sequence $0 = \lambda_0 \leq \lambda_1 \leq \lambda_2 \leq \cdots$.
	The operator $(1+\Delta)^s$ is given by
	\begin{align*}
		(1+\Delta)^s \cdot f =  \sum_{i} e_i (1+\lambda_i)^s \langle e_i , f \rangle_g,
	\end{align*}
	and so $\{ (1+ \lambda_i)^{-s/2} e_i \}_{i=1}^{\infty}$ is a Hilbert basis for $H^s(M;g)$.
	
	Let us call $e_i^{(s)} = (1+ \lambda_i)^{-s/2} e_i$.
	The embedding of $H^s(M;g)$ into $H^t(M,g)$
	is then given in terms of the respective basis elements by $e_i^{(s)} \mapsto (1+\lambda_i)^{-(s-t)/2}e_i^{(t)}$.
	As $s > t$ and $\lambda_i \to +\infty$, we see that 
	this embedding is a compact operator \cite[see Proposition 4.6]{Conway1990}.
\end{proof}

In the sequel we shall say that a wave-function $\psi \in L^2(M)$ is of class $H^s$ if there exists
a metric $g$ such that $\psi = f \sqrt{\mu_g}$, and $f \in H^s(M,g)$.\footnote{Perhaps any two Riemannian metrics on $M$
would yield the same Hilbert space $H^s(M,g)$.
If so, it would be reasonable to define a space $H^s(M)$ as a subspace of $L^2(M)$ defined independently of a metric.
We were not able to prove this statement for fractional $s$, nor were we able to find a source.
However, it is known to be true for the case where $s$ is a positive integer.} 

\section{A spectral discretization}
The goal of this section will be to produce a spectral scheme for the advection equations \eqref{eq:function pde}, \eqref{eq:density pde}, and \eqref{eq:vector field pde}
using the space $L^{2}(M)$.
In order to realize this goal we must relate functions, densities, and vector-fields to $L^{2}(M)$ in a way that can be spectrally discretized without breaking any canonical structures.

To begin, let us consider the space of continuous real-valued functions $C(M)$.
For each $f \in C(M)$ there is a unique bounded Hermetian operator, $\hat{f} : L^{2}(M) \to L^{2}(M)$ given by scalar multiplication.
That is to say $(\hat{f} \cdot \psi) (x) = f(x) \psi(x)$.
We see that the hat-map, $f \mapsto \hat{f}$, preserves the ring structure because $\widehat{f \cdot g + h} = \hat{f} \cdot \hat{g} + \hat{h}$.

Similarly, for any distribution $\rho \in C(M)'$ there is a unique operator $\hat{\rho}$ of rank 1 defined such that 
\begin{align*}
	 \int f \, \rho = \Tr ( \hat{f}^{\dagger} \cdot \hat{\rho} )
\end{align*}
for any $f \in C(M)$.
By construction, the maps $\rho \mapsto \hat{\rho}$ and $f \mapsto \hat{f}$ preserve the duality between functions and distributions.

Finally, for any vector field $X$ there is a unique (unbounded) anti-symmetric operator $\pi(X)$ on $L^{2}(M)$
given in coordinates by
\begin{align}
	\pi(X) \cdot \psi = \frac{1}{2} X^{j} \pder{\psi}{x^{j}} + \frac{1}{2} \pder{}{x^{j}} ( \psi \, X^{j} ). \label{eq:representation}
\end{align}
After some minor computations (see Appendix \ref{app:Lie} \todo{need to write an appendix} ) we see that $\pi$ is a Lie algebra morphism which sends
the Jacobi-Lie bracket to (minus) the commutator bracket.  That is to say, $\pi$ is linear and $\pi([X,Y]) = - [ \pi(X) , \pi(Y)]$.

We can now convert the evolution pdes \eqref{eq:function pde}, \eqref{eq:density pde}, and \eqref{eq:vector field pde}
into ode's of operators on $L^{2}(M)$.

\begin{thm} \label{thm:quantize}
	Let $X(t) \in \mathfrak{X}(M)$ be a time-dependent vector-field.
	Then $f$, $\rho$, and $Y$ satisfy \eqref{eq:function pde},\eqref{eq:density pde}, and \eqref{eq:vector field pde} respectively
	if and only if $\hat{f}$, $\hat{\rho}$, and $\pi(Y)$ satisfy the operator ODEs
	\begin{align}
		&\frac{d \hat{f} }{dt} + [ \hat{f} , \pi(X) ] = 0 \label{eq:quantum observable ode} \\
		&\frac{d \hat{\rho} }{dt} + [ \hat{\rho} , \pi(X) ] = 0 \label{eq:quantum density ode} \\
		&\frac{d \pi(Y) }{dt} + [ \pi(Y), \pi(X) ] = 0 \label{eq:quantum vf ode}
	\end{align}
	respectively.
\end{thm}

\begin{proof}
	Let $f$ satisfy \eqref{eq:function pde}.
	For an arbitrary $\psi \in L^{2}(M)$ we observe that $[ \hat{f} , \pi(X)] \cdot \psi$ is given in coordinates by
	\begin{align*}
		[ \hat{f} , \pi(X) ] \cdot \psi = f \left( \frac{1}{2} X^{j} \pder{\psi}{x^{j}} + \frac{1}{2} \pder{}{x^{j}} ( \psi \, X^{j} ) \right)
			- \frac{1}{2} X^{j} \pder{}{x^{j}}( f \psi)  + \frac{1}{2} \pder{}{x^{j}} (f \psi \, X^{j} )
	\end{align*}
	where we have used \eqref{eq:representation}.  Application of the product rule to all these terms then yields
	a number of cancellations and we find
	\begin{align*}
		[ \hat{f} , \pi(X) ] \cdot \psi = - X^{j} \pder{f}{x^{j}} \psi = (\partial_{t} f )\psi = \frac{d \hat{f} }{dt} \cdot \psi.
	\end{align*}
	As $\psi$ is arbitrary we have managed to show that $\hat{f}$ satisfies \eqref{eq:quantum observable ode}.
	Each line of reasoning is reversible, and so we have proven the converse as well.
	
	In order to handle densities note that $\langle f , \rho \rangle = \int_{M} f\rho$ is constant in time when $f$
	and $\rho$ satisfy \eqref{eq:function pde} and \eqref{eq:density pde} respectively.
	By the definition of $\hat{\rho}$ we observe $\Tr( \hat{f}^{\dagger} \hat{\rho}) = \langle f , \rho \rangle$.
	Thus we find
	\begin{align*}
		0 = \frac{d}{dt} \left( \Tr( \hat{f}^{\dagger} \rho ) \right) = \Tr \left( \frac{d \hat{f}^{\dagger}}{dt} \hat{\rho} + \hat{f}^{\dagger} \frac{d \hat{\rho}}{dt} \right).
	\end{align*}
	As was just shown, $\frac{d}{dt} \hat{f} =  - [\hat{f} , \pi(X) ]$ so we observe
	\begin{align*}
		0 = \Tr( - [\hat{f} , \pi(X) ]^{\dagger} \hat{\rho} + \hat{f}^{\dagger} \hat{\rho} ) = \Tr( - \pi(X)^{\dagger} \hat{f}^{\dagger} \hat{\rho} + \hat{f}^{\dagger} \pi(X)^{\dagger} \hat{\rho} + \hat{f}^{\dagger} \frac{d\hat{\rho}}{dt} ).
	\end{align*}
	Upon noting that $\pi(X)^{\dagger} = - \pi(X)$ and use of identity $\Tr( a b c) = \Tr( bc a)$ we obtain
	\begin{align*}
		0 = \Tr( \hat{f}^{\dagger}( [\hat{\rho} , \pi(X) ] + \frac{d \hat{\rho}}{dt} ) ).
	\end{align*}
	As $\hat{f}$ was chosen arbitrarily, the desired result follows.
	Again, this line of reasoning is reversible.
	
	Lastly, if $Y$ satisfies \eqref{eq:vector field pde}, then we observe 
	\begin{align*}
		\frac{d}{dt} \pi(Y) = \pi ( \partial_{t} Y ) = \pi( - [X,Y] ) = [\pi(X) , \pi(Y) ].
	\end{align*}
	where we have used the fact that $\pi$ is bracket preserving.
\end{proof}

The benefit of representing our pde's in the form \eqref{eq:quantum observable ode},\eqref{eq:quantum density ode}, and \eqref{eq:quantum vf ode}
is that we may construct discretization schemes for these equations using least squares projections.
Specifically, we have the following numerical scheme for \eqref{eq:function pde}

\begin{center}
\fbox{
	\begin{minipage}{0.8\textwidth}
		{\bf Problem:} Given $f_{0} \in C(M)$ solve \eqref{eq:function pde} on the time interval $[0,T]$.\\
		{\bf Numerical solution:}
		\begin{enumerate}
			\item Choose an $n$-dimensional subspace $V_{n} \subset H^{1+s}(M) \subset L^{2}(M)$ for $s > 0$.
			Let $i_{n}:V_{n} \to L^{2}(M)$ and $\pi_{n}: L^{2}(M) \to V_{n}$ be the canonical immersion an projection.
			\item Let ${\bf f}_{n,0} := \pi_{n} \cdot \hat{f}_{0} \cdot i_{n}$ and ${\bf X}_{n} := \pi_{n} \cdot \widehat{X} \cdot i_{n}$.
			\item Solve the finite dimensional ode $\dot{ \bf f}_{n} + [ {\bf f}_{n} , {\bf X}_{n} ] = 0$ with initial condition ${\bf f}_{n,0}$ on the interval $[0,T]$.
			\item Set $f_{n}(t) = \operatorname{arginf}_{g \in C(M)} \| \hat{g} - {\bf f}_{n} \|_{op} $ where $\| \cdot \|_{op}$ is the operator norm.
		\end{enumerate}
	\end{minipage}
}
\end{center}

The final few steps might require explanation.
It will be shown that as $n$ increases ${\bf f}_{n}(t)$ converges to the solution of \eqref{eq:quantum observable ode} with respect to the operator norm.
By theorem \eqref{thm:quantize} we know that the solution of \eqref{eq:quantum observable ode} is related to the solution of \eqref{eq:function pde}
through the map $f \mapsto \hat{f}$.
If one desires to get a function rather than an operator we must find a left inverse of the hat-map (in fact ${\bf f}_{n}$ will typically fall slightly outside the range of the hat-map).
This is what is accomplished in step (4) where an optimization problem yields a function on $M$.
We will later show that the function $f_{n}(t)$ converges to $f$ in the sup-norm.

We use roughly the same exact algorithm for advection of densities (except this time you use the dual norm to $\| \cdot \|_{op}$.
We also us the same algorithm for vector-fields, although the final inversion step dropped and we leave things in operator form.

\subsection{Convergence rates}

\section{Numerical experiments}

\section{$C^{*}$-algebraic and quantum mechanical interpretations}

\section{Conclusion}

\subsection{Future work}

\subsection{Acknowledgements}

\appendix

\section{Proof that $\pi$ is bracket preserving} \label{app:Lie}
Here we prove that $\pi$ is a Lie algebra morephism.

\bibliographystyle{amsalpha}
\bibliography{hoj.bib}
\end{document}
