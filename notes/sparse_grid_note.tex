\documentclass[12pt]{amsart}
\usepackage{amsmath,amssymb}
\usepackage{geometry} % see geometry.pdf on how to lay out the page. There's lots.
\geometry{a4paper} % or letter or a5paper or ... etc
% \geometry{landscape} % rotated page geometry
\usepackage{hyperref}
\newtheorem{thm}{Theorem}

%  TITLE, AUTHOR, DATE
\title{Open issues}
\author{Henry O. Jacobs}
\date{\today}


\begin{document}

\maketitle

\section{An issue with using sparse grids}

All methods for sparse grids use the function space
\begin{align*}
	X^{q,2} = \left\{ u \in L^q \mid \frac{\partial^{\| \mathbf{\alpha} \|_1} u }{ \partial x_1^{\alpha_1} \cdots \partial x_d^{\alpha_d}} \in L^q \quad ,\quad  \forall \alpha \ni \|\mathbf{\alpha} \|_\infty \leq 2  \right\}
\end{align*}
for $q \geq 1$.
The basic theorem for convergence on sparse grids is

\begin{thm}[Theorem 3.8 \cite{BungartzGriebel2004}]
	For $u \in X^{q,2}$ the sparse-grid interpolation, $u_n$ satisfies the error bounds
	\begin{align*}
		\| u - u_n \|_{p} = \mathcal{O}(h_n^2 \cdot n^{d-1} )
	\end{align*}
	for $p=2,\infty$
	and 
	\begin{align*}
		\| u - u_n \|_E = \mathcal{O}(h_n)
	\end{align*}
	where $h_n \sim 2^{-n}$.
\end{thm}

This rate is great.  It gives high-dimensional systems convergence rates comparable to one-dimensional systems.
However, the requirement that $u \in X^{q,2}$ is expensive to enforce in a convex solver.
There are only two cases where this constraint is a linear inequality, $q= 1$ and $q=\infty$.  Both are problematic.

In the case of $q=1$ this corresponds to $3^d$ constraints, each of which is represented by a dense row in the constraint matrix.
If we use only 3 nodes in each coordinate direction (which seems ridiculously low) then we get $6^d$ coefficients to place in $A$.
For $d=10$ this is 60 million coefficients. (beyond largest benchmark at \url{http://plato.asu.edu/ftp/lpcom.html})
For $d=15$ this is 500 billion coefficients.

In the case of $q=\infty$ we have $3^d$ constraints for each node, of which we should expect at least $2^d$ such nodes.
So this again leads to $6^d$ non-zero entries in the constraint matrix.

\section{Open issue with the Haar wavelet}
It's not clear what our convergence rate is when we use a Haar wavelet.  Our paper only addresses the case of a differentiable basis.
I think we have convergence, but we've sacrificed the exponential convergence rate for a first order or possibly slower rate. (should we consider finite differences?)

\bibliographystyle{amsalpha}
\bibliography{hoj}
\end{document}
