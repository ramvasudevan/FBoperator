\documentclass[12pt]{amsart}
\usepackage{amsmath,amssymb}
\usepackage{geometry} % see geometry.pdf on how to lay out the page. There's lots.
\geometry{a4paper} % or letter or a5paper or ... etc
% \geometry{landscape} % rotated page geometry

%  POSSIBLY USEFULE PACKAGES
%\usepackage{graphicx}
\usepackage{tensor}
%\usepackage{todonotes}

%  NEW COMMANDS
\newcommand{\pder}[2]{\ensuremath{\frac{ \partial #1}{\partial #2}}}
\newcommand{\ppder}[3]{\ensuremath{\frac{\partial^2 #1}{\partial
      #2 \partial #3} } }

%  NEW THEOREM ENVIRONMENTS
\newtheorem{thm}{Theorem}[section]
\newtheorem{prop}[thm]{Proposition}
\newtheorem{cor}[thm]{Corollary}
\newtheorem{lem}[thm]{Lemma}
\newtheorem{defn}[thm]{Definition}


%  MATH OPERATORS
\DeclareMathOperator{\Diff}{Diff}
\DeclareMathOperator{\GL}{GL}
\DeclareMathOperator{\SE}{SE}
\DeclareMathOperator{\ad}{ad}
\DeclareMathOperator{\Ad}{Ad}
\DeclareMathOperator{\Herm}{Herm}

%  TITLE, AUTHOR, DATE
\title{Notes on ROA project}
\author{Henry O. Jacobs}
\date{4th of June, 2015}


\begin{document}

\maketitle

\section{Equations of motion for dice}

\section{Constraint equation}

Let $\psi(t),\nu(t) \in L^2(M)$ for $t \in [0,T]$.
Then observe that
\begin{align*}
  \langle \psi(T) , \nu(T) \rangle_{L^2} = \langle \psi(0) , \nu(0) \rangle_{L^2} + \int_0^T \frac{d}{dt} \langle \psi(t) , \nu(t) \rangle dt \\
 = \langle \psi(0) , \nu(0) \rangle_{L^2}  + 
 \int_0^T \langle \partial_t \psi , \nu \rangle_{L^2} + \langle \psi , \partial_t \nu \rangle_{L^2} dt
\end{align*}
Let $A$ be an anti-symmetric operator on $L^2(M)$.
Then $\psi(t)$ satisfies
\begin{align}
  \partial_t \psi + A[\psi] = 0 \label{eom}
\end{align}
if and only if
\begin{align*}
  \langle \partial_t \psi + A[\psi] , \nu \rangle_{L^2} = 0
\end{align*}
for all $\nu \in L^2(M)$.
Invoking the anti-symmetry of $A$ we find that the above equation holds if and only if
\begin{align*}
  \langle \partial_t \psi , \nu \rangle =  \langle \psi , A[\nu] \rangle_{L^2} 
\end{align*}
for all $\nu$.

Thus $\psi$ satisfies \eqref{eom} if and only if
\begin{align*}
  \langle \psi(T) , \nu(T) \rangle_{L^2} = \langle \psi(0) , \nu(0) \rangle_{L^2} + \int_0^T \langle \psi , \partial_t \nu + A[\nu] \rangle_{L^2} dt
\end{align*}
for all $\nu(t) \in L^2(M)$ with $t \in [0,1]$.


In our case $A[\psi] (x) := X^i(x) \partial_i\psi(x) + \frac{1}{2} \partial_i X^i(x) \psi(x) $ and we use weak-derivatives if needed.

\section{Controls}
We wish to solve for the region of attraction under the controlled dynamics
\begin{align*}
	\dot{x} = X_0(x) + \sum_{k=1}^{c} u^k X_k(x).
\end{align*}
where $X_0,X_1,\dots,X_c$ are vector fields.
Define the unbounded Hermetian operators $\hat{H}_i \in \Herm(L^2(M) )$ by
\begin{align*}
	\hat{H}_i \cdot \psi := i \sum_{\alpha=1}^d   \frac{1}{2} \partial_\alpha X_i^\alpha  \psi + X_i^\alpha \partial_\alpha \psi,
\end{align*}
for $i=0,1,\dots,c$.
Then the dynamics of a half-density evolve under the controlled Schrodinger equation
\begin{align*}
	\dot{\psi} = i \hat{H} \cdot \psi + i \sum_{k=1}^c u^k \hat{H}_k \cdot \psi = i \hat{H}(u) \cdot \psi.
\end{align*}
where $\hat{H}(u) = \hat{H}_0 + u^k \hat{H}_k$.
Let $\pi_M : U \times M \to M$ be the cartesian projection onto $M$.
We can define the operator $\hat{J} : L^2(M) \to L^2(M \times U)$ given by $(\hat{J} \cdot \psi )(x,u)= (\hat{H}(u) \cdot \psi)(x) \sqrt{du}$
or more concisely $\hat{J} = \hat{H}(\cdot) \otimes \sqrt{u}$.
From here we may compute the dual operator $\hat{J}^\dagger : L^2( U \times M) \to L^2(M)$.
Explicitly $\hat{J}^\dagger$ takes the the form
\begin{align*}
	(\hat{J}^\dagger \cdot \phi )(x) :=  \int_U \hat{H}(u) \phi(x,u) du
\end{align*}
Inspired by Henrion-Korda we should be able to obtain the ROA by solving the primal QP
\begin{align*}
	p^* = \sup_{
		\substack{ \delta_T \otimes \psi_T - \delta_0 \otimes \psi_0 = \int_{0}^T \hat{J}^\dagger \cdot \phi dt \\ -1 \leq \psi_T \leq 1  }}
			 \| \psi \|_{L^2(M)}^2
\end{align*}
where the decision variables are $\phi \in H^1( [0,T] ; L^2(U \times M) )$, and $\psi_T  \in L^2(M)$.

\section{Discretization of vector-fields as operators}
\label{sec:discretization of vector-fields}
Let $X (x,t)= f^\alpha(x,t) \pder{}{x^\alpha}$.
We wish to discretize the operator
\begin{align*}
	\widehat{X}[\psi] = \frac{1}{2} \left(  \partial_\alpha( f^\alpha \cdot \psi) + f^\alpha \partial_\alpha \psi \right)
\end{align*}

What we could do is construct the operator $\widehat{\partial_\alpha}$
and the multiplication operator $\hat{f}^\alpha: \psi(\cdot ) \mapsto f^\alpha(\cdot) \cdot \psi(\cdot)$ given in matrix form by
\begin{align*}
	[\hat{f}^\alpha]\indices{^i_j} = \langle w_i \mid \hat{f}^\alpha \mid w_j \rangle.
\end{align*}
My hope is that the operator $\frac{1}{2} ( [\widehat{f^\alpha}]_n \cdot [\widehat{\partial_\alpha}]_n + [\widehat{\partial_\alpha}]_n \cdot [\widehat{f^\alpha}]_n)$ is a good approximation of $\widehat{X}$
when $[\widehat{f^\alpha}]_n$ a least squares projection of $\widehat{f^\alpha}$ and $[\widehat{\partial_\alpha}]_n$ is a least squares projection of $\widehat{\partial_\alpha}$ onto some finite-dimensional subspace $V_n$.  The following two (yet to be proven) lemmas would give us this result.

\begin{lem}
	Let $f \in L^{\infty}$, $n \in \mathbb{N}$, and
	$\{ e_0,e_1,\dots \}$ a Hilbert basis for $L^2$.
	Let $V_n = {\rm span}( e_0 , e_1,\dots,e_n)$ and let
	$\pi_N : L^2 \to V_n$ be the orthogonal projection.
	Then the operator $[\hat{f}]_n = \pi_n \circ \hat{f} \circ \pi_n$
	satisfies the error bound
	\begin{align*}
		\| \hat{f} - [\hat{f}]_n \|_{L^2} = \epsilon_1(n)
	\end{align*}
	for some function $\epsilon_1(n)$ which vanishes at infinity.
\end{lem}

\begin{lem}
	The operator $[\widehat{\partial_\alpha}]_n = \pi_n \circ \widehat{\partial_\alpha} \circ \pi_n$
	satisfies the error bound
	\begin{align*}
		\| \widehat{\partial_\alpha}|_{H^1} - [\widehat{\partial_\alpha}]_n \|_{L^2} = \epsilon_2(n)
	\end{align*}
	for some function $\epsilon_1(n)$ which vanishes at infinity.
\end{lem}

\begin{prop}
	The operator $\frac{1}{2} ( [\widehat{f^\alpha}]_n \cdot [\widehat{\partial_\alpha}]_n + [\widehat{\partial_\alpha}]_n \cdot [\widehat{f^\alpha}]_n)$
	approximates the operator $\widehat{X} = \frac{1}{2}( \widehat{f^\alpha}  \widehat{\partial_\alpha} +  \widehat{\partial_\alpha}  \widehat{f^\alpha} )$
	on $H^1$	with an error bound proportional to $\epsilon(n) = \max( \epsilon_1(n) , \epsilon_2(n) )$.
\end{prop}
\begin{proof}
	The proof comes down to the following computation for any two operators $A$ and $B$ with approximations $A_n$ and $B_n$.
	\begin{align*}
		\| AB - A_n B_n \| = \| AB - A_n B + A_n B- A_n B_n \| \\
		\leq \| AB - A_n B \| + \| A_nB - A_n B_n \| \\
		\leq \|B \| \|A - A_n \| + \|A_n \| \|B - B_n \|.
	\end{align*}
\end{proof}

\bibliographystyle{amsalpha}
\bibliography{/Users/hoj201/Dropbox/hoj_2014.bib}
\end{document}
