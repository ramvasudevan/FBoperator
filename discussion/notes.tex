\documentclass[12pt]{amsart}

% PACKAGES
\usepackage{amssymb}
\usepackage{todonotes}

% COMMANDS
\DeclareMathOperator{\Diff}{Diff}
\DeclareMathOperator{\Meas}{Meas}

% THEOREM ENVS
\newtheorem{lem}{Lemma}
\newtheorem{thm}{Theorem}
\newtheorem{cor}{Corollary}

\title{Notes of the detection of ROAs}
\author{Henry O. Jacobs and Ram Vasudevan}
\date{\today}

\begin{document}
\maketitle

\section{Our version of Lemma 3}

Lemma 3 of \cite{HenrionKorda2013} seems to say

\begin{lem}[Lemma 3 of \cite{HenrionKorda2013}]
  Let $(\mu_0,\mu,\mu_1)$ satisfy
  \begin{align*}
    \mu_T \otimes \delta_T - \mu_0 \otimes \delta_0 = \mathcal{L}^\dagger \mu,
  \end{align*}
  where $\mu_0$ is a measure on $X$, $\mu_T$ is a measure on $X_T$ and $\mu$ is a measure on $[0,1] \times X \times U$.
  Then there exists a family of admissible trajectories of
  \begin{align*}
    \dot{x} \in f( t , x , U),
  \end{align*}
  and a measure $\bar{\mu}$ which is advected by these trajectories, with $\bar{\mu}(i , \cdot ) = \mu_i( \cdot ) \otimes dt$ for $i=0,T$.
\end{lem}

The first lines of the proof in \cite{HenrionKorda2013} start by disintegrating $\mu$ into a tensor product $\mu(t,x,u) = \nu( u ; t,x) \otimes \bar{\mu}(x; t) \otimes dt$ where $\nu \in \Meas(U)^{ [0,T] \times X}$, $\bar{\mu} \in \Meas(X)^{[0,T]}$.
Finally, the admissible trajectories are generated by the ODE
\begin{align*}
  \dot{x} = \bar{f}(t,x) := \int_U f(t,x,u) \nu(u ; t,x).
\end{align*}

If we just incorporate this into the statement of the Lemma we get

\begin{lem}
  Let $(\mu_0,\mu,\mu_1)$ satisfy
  \begin{align*}
    \mu_T \otimes \delta_T - \mu_0 \otimes \delta_0 = \mathcal{L}^\dagger \mu,
  \end{align*}
  where $\mu_0$ is a measure on $X$, $\mu_T$ is a measure on $X_T$ and $\mu$ is a measure on $[0,1] \times X \times U$.
  Then there exists a time-dependent vector field $\bar{f}:[0,1] \times X \to TX$
and a measure $\bar{\mu}$ on $[0,1] \times X$ such that $\mathcal{L}_f [ \bar{\mu} ] = \mu_T \otimes \delta_T - \mu_0 \otimes \delta_0$.
\end{lem}

As $\bar{\mu}$ satisfies the Louiville equation, we may take it's square root to express the problem in terms of half-densities advected by the half-density advection equation.

\begin{cor}
  Let $(\mu_0,\mu,\mu_1)$ satisfy
  \begin{align*}
    \mu_T \otimes \delta_T - \mu_0 \otimes \delta_0 = \mathcal{L}^\dagger \mu,
  \end{align*}
  where $\mu_0$ is a measure on $X$, $\mu_T$ is a measure on $X_T$ and $\mu$ is a measure on $[0,1] \times X \times U$.
  Then there exists a time-dependent vector field $f:[0,1] \times X \to TX$
and a time-dependent half-density, $\psi$, on $X$ such that $\mathcal{L}_f [ \psi \otimes dt ] = \psi_T \otimes \delta_T - \psi_0 \otimes \delta_0$.
\end{cor}


\todo[inline]{Henry:  So now we form this as an optimization problem over time-dependent half-densities on $X$, and space-time dependent half-densities on $U$ (who's square produces the conditional measure $\nu$ which generates the vector field $\bar{f} = \langle \nu , f \rangle$.  Am I correct?}

\section{Forming this as an optimization problem}

We first seek to put this in the form of an optimzation problem as in Henrion-Korda.  Let $u \in \mathfrak{X}(M)$ and we wil denote the flow of $u$ by $\exp(t u)$.  Let $\mu \in \mathrm{Dens}(M)$ be the Lebesgue measure and denote the action of a diffeomorphism, $g \in \Diff(M)$ on a half-density, $\psi \in \mathcal{H}(M)$ by $g \cdot \psi$.  Consider the optimization problem on half-densities
\begin{align*}
  q^* = \inf  \left( \langle \mu^{1/2} , \psi_1 \rangle \mid 
    \begin{array}{l}
      \psi_1 = \exp (u) \cdot \psi_0 \\
      \mathrm{support}(\psi_0) = X_0
    \end{array}
  \right)
\end{align*}

We can put this in a more standard form by substituting the constraint 
``$\mathrm{support}(\psi_0) = X_0$'' with the constraint
``$\langle \phi , \left. \psi_0 \right|_{X_0^c} \rangle = 0$'' for all
$\phi \in \mathcal{H}( X_0^c )$ where $X_0^c = M \backslash X_0$.  We get the primal problem

\begin{align*}
  q^* = \inf  \left( \langle \mu^{1/2} , \psi_1 \rangle \mid 
    \begin{array}{l}
      \psi_1 = \exp (u) \cdot \psi_0 \\
      \langle \phi , \left. \psi_0 \right|_{X_0^c} \rangle = 0 , \forall \phi \in \mathcal{H}(X_0^c)
    \end{array}
  \right).
\end{align*}
Now that the problem is in the standard form and the Lagrangian dual function is
\begin{align*}
  L(\psi_0,\psi_1,\lambda) = \langle \mu^{1/2} , \psi \rangle + \langle \lambda_0 , \psi - \exp(u) \cdot \psi_0 \rangle + \sum_{ \lambda \in \mathcal{H}( X_0^c) } \langle \lambda , \left. \psi_0 \right|_{X_0^c} \rangle.
\end{align*}
Unfortunately, this Lagrangian is linear in $\psi_1$ and $\psi_0$ and so the Lagrangian dual function
\begin{align*}
  g(\lambda) := \inf_{\psi_0,\psi_1}( L(\psi_0,\psi_1,\lambda) ) = -\infty.
\end{align*}
So that the dual problem
\begin{align*}
  d^* = \sup_\lambda( g(\lambda) )
\end{align*}
is nonsense.

\bibliographystyle{amsalpha}
\bibliography{./hoj_2014}

\end{document}